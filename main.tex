\documentclass[a4paper, 12pt]{article}
\usepackage[left=2.5cm,right=2.5cm,
            top=2.5cm,bottom=2.5cm,
            bindingoffset=0cm]{geometry}

% useful package for XeLaTex
\usepackage{xltxtra}

% some specifications for math text
\defaultfontfeatures{Ligatures=TeX,Mapping=tex-text}
\usepackage[LGRgreek,noendash]{mathastext}
\usepackage{amsmath}

% setting default fonts
\usepackage{fontspec}
\setmainfont{Brill-Roman.ttf}[
 ItalicFont = Brill-Italic.ttf,
 BoldFont = Brill-Bold.ttf,
 BoldItalicFont = Brill-BoldItalic.ttf]
\setsansfont{FreeSans}
\setmonofont{FreeMono}

% ipa symbols if needed
\usepackage{tipa}


% babel
\usepackage[english]{babel}


% tikz for images and forest for linguistic trees
\usepackage{tikz}
\usetikzlibrary{calc}
\usetikzlibrary{positioning}
\usepackage[linguistics]{forest}

% library for nice arrows
\usetikzlibrary{paths.ortho}


% styles enumeration
\usepackage[shortlabels]{enumitem}


% gloss formatting
\usepackage{expex}
\lingset{everygla=\normalfont, belowglpreambleskip=-6pt, aboveglftskip=-6pt, belowexskip=0pt, aboveexskip=0pt, glhangstyle=none, extraglskip=!.25ex}


% glossing
\usepackage{leipzig}
% automatic small caps
\renewcommand{\leipzigfont}[1]{\textsc{#1}}
% input file with glosses for leipzig package
\newleipzig{Coref}{coref}{coreferent}
\newleipzig{Npst}{npst}{non-past}
\newleipzig{Ep}{ep}{epenthesis}
\newleipzig{Inch}{inch}{inchoative}
\newleipzig{Ss}{ss}{same subject}
\newleipzig{Todp}{tod.p}{today's past tense}
\newleipzig{S}{s}{singular}
\renewleipzig{Ptcp}{ptc}{participle}
\newleipzig{Nfut}{nfut}{non-future}
\newleipzig{Dir}{dir}{directive}
\renewleipzig{Incl}{inc}{inclusive}
\renewleipzig{Excl}{exc}{exclusive}
\newleipzig{Prol}{prol}{prolative}
\newleipzig{Dloc}{dloc}{directive-locative}
\newleipzig{Add}{add}{additive}
\newleipzig{Freq}{freq}{frequentative}
\newleipzig{Dim}{dim}{diminutive}
\newleipzig{ConvAnt}{convant}{~~~~~~anterior converb}
\newleipzig{Sup}{sup}{supine}
\newleipzig{Pr}{pr}{~}
\newleipzig{And}{and}{~}
\newleipzig{Impf}{impf}{imperfective}
\newleipzig{Nsg}{nsg}{non-singular}
\newcommand{\Fnsg}{\First{}\Nsg{}}
\newleipzig{A}{a}{absolutive}
\newleipzig{E}{e}{ergative}
\newleipzig{Nonactive}{nonactive}{non-active}
\newleipzig{Clitic}{clitic}{clitic}
\newleipzig{Nw}{nw}{nonwitnessed}
\newleipzig{Wp}{wp}{witnessed past tense}
\newleipzig{V}{v}{\textsc{v} class in Ingush}
\newleipzig{J}{j}{\textsc{j} class in Ingush}
\newleipzig{I}{i}{first gender in Archi}
\newleipzig{II}{ii}{second gender in Archi}
\newleipzig{III}{iii}{third gender in Archi}
\newleipzig{IV}{iv}{fourth gender in Archi}
\newleipzig{Evid}{evid}{evidentiality}
\newleipzig{Int}{int}{intensifier particle}
\newleipzig{Atr}{atr}{attributivizer}
\newleipzig{Ill}{ill}{illative}
\newleipzig{Trans}{trans}{transitional}
\newleipzig{Compl}{compl}{completion}
\newleipzig{In}{in}{inessive}
\newleipzig{Fst}{fst}{false start}
% \newleipzig{}{}{}
% \newleipzig{}{}{}
% \newleipzig{}{}{}
% \newleipzig{}{}{}
% list of abbreviations
\usepackage{glossaries}
\makeglossaries


% adding appendix
\usepackage[toc,page]{appendix}


% styling hyperlinks
\usepackage{hyperref}
\hypersetup{
 colorlinks,
 citecolor=[RGB]{34, 37, 117},
 filecolor=black,
 linkcolor=[RGB]{34, 37, 117},
 urlcolor=[RGB]{34, 37, 117},
 pdfborder={0 0 0}
}

% citation format
\usepackage[parentracker=true,
backend=biber,natbib,
hyperref=true,
bibencoding=utf8,
maxcitenames=2,
style=biblatex-gl,
citestyle=gl-authoryear-comp
]{biblatex}

% cite year command
\DeclareCiteCommand{\citeyear}
    {}
    {\bibhyperref{\printfield{year}}\bibhyperref{\printfield{extradate}}}
    {\multicitedelim}
    {}

% cite year in parentheses
\DeclareCiteCommand{\citeyearpar}
    {}
    {\mkbibparens{\bibhyperref{\printfield{year}}\bibhyperref{\printfield{extradate}}}}
    {\multicitedelim}
    {}

% citation in parentheses
\newcommand{\parref}[1]{(\ref*{#1})}


% quotation
\usepackage{csquotes}


% matrices
\usepackage{array}
% vertical spacing between rows in arrays
\renewcommand{\arraystretch}{1.25}


% adding multicolumn mode
\usepackage{multicol}
\setlength{\columnsep}{-0.5cm}


% nice and long tables
\usepackage{booktabs}
\usepackage{longtable}

% captions for subfigures and subtables
\usepackage{subcaption}

% default bib resource
\addbibresource{bib.bib}


% default spacing set to 1.5, no parindent,
% spacing after paragraphs is 6pt
\usepackage{setspace}
\onehalfspacing
\setlength{\parindent}{0pt}
\setlength{\parskip}{6pt}

% list of abbreviation formating
\renewcommand{\glossarypreamble}{\hfill\begin{minipage}{\textwidth}}
\renewcommand{\glossarypostamble}{\end{minipage}\hfill}

% special apostrophe
\renewcommand{\'}{\textquotesingle}

\usepackage{multirow}

\singlespacing

\author{Anton Buzanov}
\title{Possessive Marking: Dependent and Non-dependent marking}
\date{\today}

\begin{document}
\thispagestyle{empty}	
\begin{center}
	{\fontsize{14pt}{1}\selectfont \textbf{ПРАВИТЕЛЬСТВО РОССИЙСКОЙ ФЕДЕРАЦИИ\\[0.5em]Федеральное государственное автономное образовательное учреждение высшего образования\vspace{2em}\\Национальный исследовательский университет\\[0.5em]«Высшая школа экономики»} \\[2em]}
	{\hfill \fontsize{14pt}{1}\selectfont Факультет гуманитарных наук \\[0.5em]}
	{\hfill \fontsize{14pt}{1}\selectfont Образовательная программа \\[0.5em]}
	{\hfill \fontsize{14pt}{1}\selectfont <<Фундаментальная и компьютерная лингвистика>> \\[3em]}
	{\fontsize{14pt}{1}\selectfont Бузанов Антон Олегович \\[2em]}
	{\fontsize{14pt}{1}\selectfont \textbf{Теория и типология рефлексивного посессивного согласования} \\[0.5em]}
	{\fontsize{14pt}{1}\selectfont \textbf{Theory and Typology of Reflexive Possessive Agreement} \\[2em]}
	{\fontsize{14pt}{1}\selectfont Выпускная квалификационная работа\\[2em]}
	\fontsize{14pt}{1} \selectfont{
		\begin{table}[h!]
			\begin{tabular}{l l l} 
				\parbox{7cm}{\raggedright Академический руководитель образовательной программы} &\hspace{1.5cm} & Научный руководитель  \\ 
				\parbox{7cm}{канд. филологических наук, доц.} & & доктор философии по лингвистике, доц. \hspace{1cm} \\
				Ю.~А.~Ландер & & \parbox[l]{6cm} {\tolerance-1 П.~В.~Руднев \hspace{1cm}} \\
				& & \\
				\multicolumn{3}{l}{<<~~~~~~~>>~\_\_\_\_\_\_\_\_\_\_\_\_\_\_\_\_\_~2022г.}
			\end{tabular} 
		\end{table}
	}
	\vspace*{\fill}
	{\fontsize{14pt}{1}\selectfont Москва, 2022~г.}
	
\end{center}

\section{Introduction}

This thesis is dedicated to possessive constructions. Within this study, possessive constructions are defined as comprising two noun phrases: the possessor and the possessum. Their primary function is to convey possessive relationships, which are broadly interpreted. Among these relationships, ownership stands out as the most common.

To illustrate these concepts, I provide several examples of possessive constructions in (\ref{primary_examples}). These examples are to clarify the core focus of the present study. Furthermore, they highlight the ways in which possessive relationships can be marked within noun phrases. For instance, while the Russian example indicates possession solely through the genitive case on the dependent, the Even example employs a possessive suffix on the head to convey the possessive relationship.

\pex
\label{primary_examples}
\a
\begingl
\glpreamble Russian (Indo-European)//
\gla dom-\o{} otc-a//
\glb house-\Nom.\Sg{} father-\Gen.\Sg{}//
\glft `father\'s house'//
\endgl 

\a
\begingl
\glpreamble Even (Tungusic)//
\gla etiken  ǯu-n//
\glb old.man house-\Poss.\Tsg{}//
\glft `old man's house'//
\endgl 
\xe

Various types of possessive constructions can be identified. For instance, in Russian, there exists another construction primarily used to express possessive relationships, as shown in (\ref{possessive_adjective}). Forms such as \textit{mam-in-a} are commonly referred to as ``possessive adjectives" (\textit{pritjažatel'nye prilagatel'nye} in Russian). This terminology suggests that possessors in these constructions behave more like adjectives than proper nouns. They inflect in oblique cases as adjectives and cannot be modified by other adjectives.

Constructions where either the possessor or the possessum deviates from typical noun behavior were excluded from the analysis. This exclusion criterion is relevant primarily in languages with a clear distinction between nouns and adjectives. The focus of this study remains exclusively on the interaction between noun phrases. Thus, the behaviour of external possessors is also beyond the scope of this study.

\ex
\label{possessive_adjective}
\begingl
\glpreamble Russian (Indo-European)//
\gla mam-in-a kniga//
\glb mother-\Poss.\Adj-\Nom.\Sg.\F{} book//
\glft `mother\'s book'//
\endgl
\xe

The distinction between alienable and inalienable possession \citep{nichols1988alienable,alexiadou2003some} is not the central focus of this thesis. Both types of constructions are treated with equal importance.

The primary objective of this study is to develop a typology for possessive marking across various language types. Specifically, the focus lies on morphosyntactic patterns such as the locus of marking, obligatoriness, and agreement. Although a general typology of locus marking was developed by \cite{nichols1986head}, I believe that this typology is insufficient and does not capture some differences and similarities among languages. There was another attempt to classify possessive constructions by \cite{croft2002typology}. However, he admits that the typology presented there serves only illustrative purposes and does not claim comprehensiveness.

The thesis is structured as follows:

Chapter \ref{litreview} provides an overview of previous approaches to describing possessive constructions. Chapter \ref{problems} identifies the shortcomings of these approaches and presents sporadic data that highlights the missed important generalizations. In that chapter, I introduce my own classification system for languages based on how they express possessive relations.

Chapter \ref{sample} discusses different approaches to creating linguistic samples. In developing a sample for the current study, I specified genera for approximately half of the world's languages. This detailed classification allowed me to include a broader and more diverse range of languages in the sample. Consequently, languages that might have been previously overlooked in typological studies were incorporated.

Chapter \ref{data} presents data from various languages to support the plausibility of my classification system over previous ones. Within this chapter, I highlight the differences between the dependent and non-dependent marking within noun phrases.

Chapter \ref{asymmetries} articulates the asymmetries between dependent and non-dependent possessive marking that I identified during my study. 

Finally, Chapter \ref{conclusion} draws conclusions based on the findings presented in the preceding chapters.

\section{Previous Studies}
\label{litreview}

Possessive constructions have been described extensively for individual languages \citep{bugenhagen1986possession,lehmann1998possession,krasnoukhova2011attributive}, as well as in cross-linguistic studies \citep{aikhenvald2013possession,nichols1986head,nichols1988alienable,croft2002typology,}. However, it is noteworthy that most cross-linguistic investigations have focused primarily on alienability, existential, or locative possessives, or solely on possessive constructions marked on the dependent.

\subsection{Registration and Indexation}
\label{sec:registration_indexation}

One aspect that I will address is the distinction between indexation and registration \citep{nichols1992linguistic}. Indexing markers reflect certain $\phi$-features of another part of the construction, while registering markers simply indicate the presence of another part. Originally, this notion aimed to differentiate between two types of head-marking possessive constructions. However, \cite{lander2020head} extended this perspective to encompass dependent marking as well.

Similar concepts, albeit under different names, have been discussed by several other authors, including Creissels (2017) and Plank (1995).

In a related vein, \cite{duguine2008structural} explores this distinction in various languages. She demonstrates that possessors that trigger agreement can be pro-dropped and extracted, whereas possessors that do not trigger agreement cannot be treated in the same manner.

Although these concepts are mentioned by various authors within different frameworks, there is, as far as I am aware, no unified account or set of generalizations based on the nature of a marker. In this study, I aim to systematically explore how markers differ in this aspect, providing a thorough analysis and potential generalizations that could apply across languages.

\subsection{Locus of Marking in Possessive Noun Phrases}

The literature on the locus of marking, particularly as discussed by \cite{nichols1986head} and later scholars, provides perhaps the most theoretically neutral typology of possessive noun phrases. \cite[57]{nichols1986head} argues that the locus of marking does not signify any distinction in the syntactic relation, which remains consistent across different marking patterns.

Based on the specific locus of marking within possessive noun phrases, languages can be distinctly categorized into several groups \citep{nichols_locus_2013,van2016locus,van2016grammaticalization, lander2020head}.

The types of possessive marking within a possessive noun phrase, as distinguished by \cite{nichols_locus_2013}, along with the respective numbers of languages for each type in brackets, are as follows.

\begin{enumerate}
    \item Head marking (78)
    \item Dependent marking (98)
    \item Double marking (22)
    \item No marking (32)
    \item Other (6)
\end{enumerate}

However, this classification seems problematic for several reasons. A later revision of this categorization was presented in \cite{lander2020head}. In this revised classification, \cite{lander2020head} introduced D- and C-marking, corresponding to dependent marking and anywhere-but-dependent marking. This distinction enables the classification of various cases initially placed in the \textit{Other} category.

To better understand the labels employed by \cite{nichols_locus_2013}, and where necessary, incorporating insights from \cite{lander2020head}, I will describe their meanings.

\subsubsection{Head marking}

\begin{quote}
    ``In these [\ref{headmarking}] examples the possessed noun (the head) agrees in person and number with the possessor noun, the most common pattern for head-marked noun phrases. Agreement in gender is also fairly common. A few languages have a non-agreeing marker on the head noun. In Fijian, a possessive affix \textit{-i} marks possessed nouns; that it does not vary for person or number of the possessor.'' \citep{nichols_locus_2013}
\end{quote}

In the examples (\ref{headmarking}), the possessed entity receives a distinctive marker that serves to signify (i) its role as the possessed item and/or (ii) $\phi$-features, encompassing person, number, and gender, of the possessor. If this marker indicates both (i) and (ii), it is recognized as the equivalent of agreement between a verb and its argument, albeit occurring within the nominal domain. In the framework of \cite{lander2020head}, the representation of features related to the possessed entity is termed \textit{indexation} in contrast to \textit{registration}.

\pex\label{headmarking}
\a
\begingl
\glpreamble Acoma (Keresan; \citealt[177]{Miller-1965} from \citealt{nichols_locus_2013}) //
\gla s'adyúm'ə gâam'a//
\glb \Fsg.brother \Tsg.house//
\glft `my brother's house' (lit. `my-brother his-house')//
\endgl
\a
\begingl
\glpreamble Fijian (Austronesian; \citealt[36]{Dixon-1988} from \citealt{nichols_locus_2013}) //
\gla a liga-i ‘eirau//
\glb \Art{} hand-\Poss{} \Fdu.\Excl{}//
\glft `our hand(s)'//
\endgl

\xe 

Notably, the possessor in this context remains morphologically unmarked.

While \cite{nichols_locus_2013} categorize non-agreeing possessive markers, they do not explore the specific characteristics of these markers in detail. Instead, their aim is to group non-agreeing markers together with agreeing markers. Although they discuss the rationale behind this approach, it is important to investigate and find some generalizations based on the distinctions between these types of markers. This study will seek to identify and analyze these generalizations, providing a more nuanced understanding of possessive marking in various languages.


\subsubsection{Dependent marking}

\begin{quote}
    ``In the following examples from Chechen (Nakh-Daghestanian), the possessor noun is in the genitive case.'' \citep{nichols_locus_2013}
\end{quote}


\pex\label{dependentmarking}
\glpreamble Chechen (\citealt{nichols_locus_2013}) //
\a
\begingl
\gla loem-an k'orni//
\glb lion-\Gen{} baby.animal//
\glft `lion cub’, `lion's cub’ (lit. `of-lion cub’)//
\endgl
\a
\begingl
\gla mashien-an maax//
\glb car-\Gen{} price//
\glft `the price of a car' (lit. `of-car price')// 
\endgl
\xe

In (\ref{dependentmarking}), only the possessor is marked with a special marker, while the possessum would be marked with an externally governed case, although it is not evident from the examples.

\subsubsection{Double marking}

\cite{nichols_locus_2013} do not provide specific commentary on double-marking constructions. Essentially, these constructions involve a combination of the two preceding types, where both the possessor and the possessum are marked with distinctive indicators that signify (i) their respective roles and/or (ii) the features of the other (\ref{doublemarking}).

\ex
\label{doublemarking}
\begingl
\glpreamble Southern Sierra Miwok (Miwok-Costanoan; \citealt[133]{Broadbent-1964})//
\gla cuku-ŋ hu:ki-ʔ-hy://
\glb dog-\Gen{} tail-\Tsg//
\glft `dog's tail’ (lit. `of-dog its-tail’) //
\endgl
\xe

In example (\ref{doublemarking}), both the genitive marker \textit{-ŋ} on the possessor and the possessive suffix \textit{-hy:} on the possessum are present.

\subsubsection{No marking}

No special comments are made on no-marking constructions by \cite{nichols_locus_2013}.

\ex
\begingl
\glpreamble Asmat (Asmat family; \citealt[136, 133]{Voorhoeve-1965b} from \citealt{nichols_locus_2013}//
\gla no cem//
\glb \Fsg{} house//
\glft `my house' (lit. `I house' or `me house')//
\endgl
\xe

Here, the possessor \textit{no} and the possessum \textit{cem} are morphologically unmarked. Although it is not discussed by \cite{nichols_locus_2013}, it is worth mentioning that Asmat lack case marking at all. 

\subsubsection{Key Principles of \citeauthor{nichols_locus_2013}'s \citeyear{nichols_locus_2013} Typology}

The classification by \cite{nichols_locus_2013} involves selecting one construction per language, aiming for minimal restrictions within each. They explicitly exclude constructions that do not allow overtly expressed possessors.

Additionally, they argue that morphologically unmarked possessors in languages with cases behave differently from their marked counterparts in other languages. They focus solely on an open-class set of possessors, excluding consideration of personal pronouns.

On the other hand, \cite{lander2020head}, inspired by previous work, introduce and explore the concepts of \textit{registration} and \textit{indexation}. While these concepts are not explicitly addressed in \citeauthor{nichols_locus_2013}'s \citeyear{nichols_locus_2013} typology, they have significance for understanding possessive marking.

\subsection{\citeauthor{croft2002typology}'s \citeyear{croft2002typology} Typology of Possessive Constructions}

\cite[31--42]{croft2002typology} distinguishes several groups of strategies in possessive marking. I will address each type in the following sections.

\subsubsection{Simple Strategies: Juxtaposition, Concatenation, Fusion}

According to \cite{croft2002typology}, simple strategies involve the combination of the possessor and possessum without any special marking. This combination can result in \textbf{juxtaposition} when no part of the construction is phonologically affected by the other (\ref{ex:croft_juxt}). When the possessor behaves as a clitic, it is not phonologically independent and is attached to the possessum, termed as \textbf{concatenation} (\ref{ex:croft_concat}). Another possibility is the full \textbf{fusion} of a possessor and possessum, resulting in an indivisible word form \ref{ex:croft_fusion}.

\pex
\a\label{ex:croft_juxt}
\begingl
\glpreamble Kobon//
\gla Dumnab ram//
\glb Dumnab house//
\glft `Dumnab's house'//
\endgl

\a\label{ex:croft_concat}
\begingl
\glpreamble Tigre//
\gla səʔli-hom//
\glb photograph-3sG//
\glft `his photograph'//
\endgl

\a\label{ex:croft_fusion}
\begingl
\glpreamble Lakhota//
\gla ina/nihų/hųku //
\glft `my mother/your mother/his, her mother'//
\endgl

\xe

\cite{croft2002typology} argues that concatenation and fusion arise diachronically from juxtaposition. He observes that while the fusion strategy is rare, it is encountered with basic kin terms and pronominal possessors. Moreover, concatenation is also mostly found with pronominal possessors, however some languages allow nominal possessor to form a compound-like structure with the possessums.

\subsubsection{Relational Strategies: Case Affixes and Adpositions}

In \cite{croft2002typology}, only case affixes and adpositions are explicitly listed as relational strategies (as in \ref{ex:croft_relational}). In other words, these strategies are dependent marked. However, based on his comparison of relational strategies with indexical strategies, it seems that he refers to the same concept as registration strategies described in Section \ref{sec:registration_indexation}.

\pex
\label{ex:croft_relational}
\a
\begingl
\glpreamble Russian//
\gla kniga Ivan-a//
\glb book Ivan-\Gen//
\glft `Ivan's book'//
\endgl 

\a
\begingl
\glpreamble Bulgarian//
\gla nova-ta kniga na majka mi//
\glb new-the book of mother my//
\glft `my mother's new book' //
\endgl

\xe

In his definition, \cite{croft2002typology} combines two important properties of case marking: their occurrence on dependents and their invariability. It would be more appropriate to maintain the label of relational strategies, but to avoid narrowing down the list of possible relational strategies to just case markers.


\subsubsection{Indexical Strategies: Person Indexation and Non-person Indexation)}

In \cite{croft2002typology}, indexical strategies are mostly discussed using data from constructions other than possessives. However, some examples are provided where person indexation is illustrated by the indexation of a possessed noun with its possessor, as in (\ref{ex:croft_mam}).

\ex\label{ex:croft_mam}
\begingl
\glpreamble Mam//
\gla t-kamb` meeb`a//
\glb \Tsg-prize orphan//
\glft `[the] orphan's prize'//
\endgl
\xe

As an example of non-person indexation within a possessive domain, \cite{croft2002typology} examines Russian possessive pronouns, which are said to agree in gender with the possessum, as exemplified in \ref{ex:croft_russian}.

\ex\label{ex:croft_russian}
\begingl
\glpreamble Russian//
\gla mo-ja knig-a//
\glb my-\F\Sg{} book-\F\Sg{} //
\glft `my book'//
\endgl
\xe

It is important to note two points here. Firstly, the glossing in \ref{ex:croft_russian} is incorrect, as the suffix \textit{-a} does not carry information about gender; the gender value of a noun is determined lexically. Secondly, this indexing does not necessarily serve the purpose of expressing the possessive relation. In Russian, all prenominal pronouns, including demonstratives, agree in gender with their heads. Therefore, it might be more appropriate to consider this as non-person indexation in the domain of nominal concord, rather than within a possessive domain.

\subsubsection{Classifier Strategies}

A class of overt coded dependencies whose status as indexical or relational is problematic for \cite{croft2002typology} are classifiers. Classifiers are utilized in various grammatical constructions, including possessive constructions. The possessive classifier denotes a property of the possessed object. Example \ref{ex:classifier} presents a possessive construction in Kosraean with the classifier for plants, where the possessor is expressed only as a person suffix on the classifier.

\ex\label{ex:classifier}
\begingl
\glpreamble Kosraean//
\gla mos sʌnʌ-k //
\glb breadfruit \Clfplant-\Fsg //
\glft `my breadfruit tree' //
\endgl
\xe

Describing possessive classifiers in terms of head or dependent marking proves challenging. On one hand, in example (\ref{ex:classifier}), the classifier appears alongside the possessor suffix, suggesting an association with dependency marking. On the other hand, the possessor suffix itself serves as an instance of indexation. I will revisit this issue later in Chapter \ref{chapter:possessive_systems}.


\subsubsection{Linker Strategies}

By linkers, \cite{croft2002typology} refers to invariant elements that are difficult to describe as associated with a possessor or possessum. However, for some elements, one can infer whether they are associated with a dependent or not, based on their prosodic behavior. For example, he mentions the English possessive \textit{'s}, which cannot be treated as a case marker since English lacks other cases, however it functions as a clitic on a possessor.

\ex\label{ex:arabic}
\begingl
\gla el-ḥōš mtʿ-i //
\glb the-house \Poss.\Lnk{}-my //
\glft `my house' //
\endgl
\xe

I argue that this definition of linkers covers multiple different scenarios that must be distinguished. Some linkers allow possessive indexing suffixes to attach, as shown in example \ref{ex:arabic}. Linkers that cannot be inflected by person indexing markers are more similar to relational strategies, while linkers that can be inflected are more similar to classifier strategies.


\subsubsection{Special Form Strategies}
 
\cite{croft2002typology} argues that the morpheme within a construction may undergo fusion with either element, rendering it no longer analyzable. This results in the emergence of a specialized form of the element tailored specifically to the construction. An example of a specialized possessor form is commonly observed with independent pronominal possessors, as exemplified in Yoruba (\ref{ex:yoruba})

\ex
\label{ex:yoruba}
\begingl
\glpreamble Yoruba//
\gla ilé wa //
\glb house our //
\glft `our house' //
\endgl
\xe

Similarly, a specialized possessum form is evident in the construct form found in Semitic languages, such as Syrian Arabic (\ref{ex:syrian}):

\ex
\label{ex:syrian}
\begingl
\gla Passet haz-zalame //
\glb story:\Const{} that-fellow //
\glft `that fellow's story' //
\endgl
\xe

These specialized forms may either be suppletive with their paradigmatically contrasting forms (as in Yoruba) or simply represent a morphologically irregular alternative form (as in Syrian Arabic).


\subsection{Summary on Previous Approaches}

\citeauthor{nichols_locus_2013}' \citeyear{nichols_locus_2013} typology provides a foundational framework for understanding possessive marking across languages. It focuses on the locus of marking. However, it falls short in capturing certain nuances and similarities among languages, particularly in fully addressing the specific characteristics of possessive markers. Nichols' approach does not allow to describe variance even within one construction.

In contrast, Croft's typology delves deeper into the diverse strategies for possessive marking, including concatenation and fusion, as well as indexical and relational strategies. By examining examples from various languages, Croft sheds light on the nature of possessive marking systems. Nonetheless, Croft's typology may not provide a comprehensive coverage of all possessive marking strategies across languages, and some concepts and definitions may be ambiguous or challenging to apply universally.

Integrating aspects from both typologies may lead to a more nuanced and comprehensive understanding of possessive marking systems across different languages. In the next chapter, I will identify particular shortcomings that arise in both approaches.

\section{Problems with Previous Approaches}

In the preceding chapter, I outlined the main issues with two approaches to possessive constructions. In this chapter, I aim to comprehensively catalog all the shortcomings identified and offer illustrative language material to support my arguments.

\subsection{\citeauthor{nichols_locus_2013}'s \citeyear{nichols_locus_2013} Typology}

In this section, I will compare my observations directly to the locus-of-marking framework to enrich its scope. It is worth noting that my study does not concern itself with semantics or alienability, rendering typologies based on these factors irrelevant to my research.

I contend that the assertions made by \cite{nichols_locus_2013} are erroneous. Firstly, there is no necessity to exclude constructions lacking an overt possessor; in fact, their inclusion is essential for establishing a comprehensive typology of locus marking. Secondly, the postulation of a distinction between morphologically marked possessors, such as those bearing the genitive case, and morphologically unmarked genitives is unnecessary.

\subsubsection{Null Possessors}

I argue that what is called by \cite{nichols_locus_2013,lander2020head} a head-marking strategy is not the proper head-marking strategy but rather a double marking one.

The ideal example of head marking is that presented in (\ref{mongolian}).

\pex\label{mongolian}
\glpreamble Mongolian (Mongolic) from \cite{janhunen_mongolian_2012}//
\a\label{mosthead}
\begingl
\gla (*min-ii) duu-men'//
\glb \Fsg-\Gen{} younger.brother-\Poss.\First//
\glft `my younger brother'//
\endgl
\a
\begingl\label{mostdependent}
\gla min-ii eej(-*men')//
\glb \Fsg-\Gen{} mother-\Poss.\First//
\glft `my mother'//
\endgl
\xe

In (\ref{mongolian}), the simultaneous use of both genitive and possessive marking is precluded. This restriction can be elucidated by tracing the grammaticalization trajectory of possessive markers, originally postposed clitics derived from personal pronouns \citep{brosig2018}. In the current state, these markers are fully integrated phonologically, displaying no distinguishing features from regular affixes in Mongolian \citep[137]{janhunen_mongolian_2012}.

Example (\ref{mosthead}) exemplifies the perfect head-marking pattern, since there is no dependent to be marked. Conversely, example (\ref{mostdependent}) illustrates the ideal dependent-marking pattern.

According to the classification by \cite{nichols_locus_2013}, Mongolian is placed in the category of languages exhibiting dependent marking in possessive phrases. This is due to the fact that \cite{nichols_locus_2013} exclude constructions without overtly expressed possessors. However, it is evident to me that Mongolian must differ in this aspect from languages like Russian, a prototypical dependent marking language.

\subsubsection{Unmarked vs Marked Possessors}

\label{subsec:marking}

Classic instances of head-marking possessors include those in (\ref{headmarking}) or the one referenced in (\ref{hung}). In such examples, the possessor remains morphologically unmarked.

\ex\label{hung}
\begingl
\glpreamble Hungarian adapted from \cite[263]{szabolcsi1981possessive}//
    \gla az én kar-ja-i-m//
    \glb the I arm-\Poss-\Pl-\Fsg//
    \glft `my arms'//
\endgl
\xe

Now, I want to draw the attention to the data of Even (Tungusic) cited in (\ref{even_basic_examples}).

\pex\label{even_basic_examples}
\glpreamble Even (Tungusic), field data//
\a
\begingl
\gla bi əm-ni-wu//
\glb I.\Nom{} come-\Pst-\Fsg//
\glft `I came.'//
\endgl

\a
\begingl
\gla min ǯu-wu//
\glb I.\Obl{} house-\Poss.\Fsg//
\glft `my house'//
\endgl

\a
\begingl
\gla etiken əm-ni-n//
\glb old.man come-\Pst-\Tsg//
\glft `The old man came.'//
\endgl

\a
\begingl
\gla etiken ǯu-n//
\glb old.man house-\Poss.\Tsg//
\glft `old man's house'//
\endgl
\xe

These examples bear a resemblance to Hungarian in that ordinary nouns do not exhibit special genitive marking. However, a distinction arises between Even and Hungarian concerning possessive forms with pronouns. Notably, Even demonstrates a distinction from Hungarian by featuring distinct possessive forms for pronouns.

While it may initially seem that \cite{nichols_locus_2013} focus exclusively  on constructions with nominal possessors, Evenki, which is related to and behaves similarly to Even in this particular respect, is categorized as a \textit{Double-marking} language while Hungarian is categorized as a \textit{Head-marking} one. The primary rationale for this classification appears to be the presence of special possessive pronouns in Evenki. In the possessor position, both Evenki and Even employ an oblique stem, aligning with the direct stem for nouns and diverging from it for pronouns.

Indeed, when considering only nouns, Hungarian and Evenki appear similar, as both languages employ morphologically unmarked forms for possessors. However, despite this similarity, they are classified differently. An additional factor contributing to this discrepancy could be the presence of an obsolete genitive marker that occasionally appears on possessors in Evenki. This historical remnant might influence the classification of Evenki, leading to its categorization distinct from Hungarian despite their superficial similarity in possessive marking.

\subsubsection{Solutions}

I believe that the aforementioned problems with the classification of Mongolian, Hungarian and Evenki straightforwardly follows from wrong assumptions made by \cite{nichols_locus_2013}.

I contend that classifying languages or constructions as a whole is an inaccurate approach. A more effective method involves employing a bottom-up strategy, initially categorizing markers as either C- or D-marking \citep{lander2020head}, and subsequently classifying a language as a collection of pairs of these markers. For instance, in this framework, Russian would be characterized as a language featuring several D-markers, such as \Gen{} and possessive forms of (pro)nouns, while lacking any C-markers. This approach resembles the earlier concept of a dependent-marking language but offers a more flexible and nuanced classification.

In my second argument, I posit that treating languages with unmarked possessors differently from those with marked possessors is a priori an inaccurate approach. It is crucial to distinguish between zero marking and the absence of marking. The concept of the locus of marking presupposes the existence of a head and a dependent, with some form of relationship between the two. I propose that in languages featuring a grammaticalized case system, the possessor is indeed marked, but the marker itself is represented by a morphological zero.

This perspective might be seen as controversial, yet several arguments support this viewpoint. Consider languages like Even and Evenki, where a position is marked with the genitive, but only for personal pronouns. In such instances, it is reasonable to assume that for nouns the marking is also present, albeit in the form of zero. Moreover, adpositions in Even and Evenki also govern this unmarked form.

In the second argument, consider languages with declension classes where certain classes exhibit a syncretism between the nominative and genitive forms, as seen in Latin with \textit{avis} meaning 'bird.\Nom{} / bird.\Gen{}'. In such cases, determining whether these forms are marked or unmarked becomes ambiguous. On the one hand, the nominative is typically considered an unmarked case, even though it is morphologically marked. From a paradigmatic perspective, it might be more appropriate to label these forms as marked with the genitive rather than with the nominative.

In a language where nouns lack dedicated genitive forms and the form coinciding with the nominative is employed in possessor contexts, this shared nominative/genitive form is morphologically unmarked. This observation supports the idea of unmarked case. Essentially, it is a matter of cross-linguistic variation which form is used as a default nominal dependent.

I argue that one must consider morphologically unmarked possessors in languages that have cases to be an instance of head-marking / double-marking strategy. In the same way, we consider unmakred nominative with agreeing verb to present double-marking strategy at the clause level.


\subsection{\citeauthor{croft2002typology}'s \citeyear{croft2002typology} Typology}

The typology presented by \cite{croft2002typology} aims to illustrate different marking patterns among languages rather than covering all linguistic variance. Thus, this approach has several limitations. Firstly, it does not explicitly address the locus of marking, resulting in the classification of possessive constructions based on different characteristics that are not clearly specified. This makes it difficult to treat concatenation (indexing an obligatorily unexpressed possessor on the possessum) and person indexation in a similar manner, potentially overlooking important similarities between these marking patterns.

Moreover, \cite{croft2002typology} overlook the variation within possessive constructions, particularly by ignoring double-marked constructions. His focus primarily lies on constructions involving a single element that marks the possessive relation, limiting the scope of analysis.

Another issue is the lack of consideration for the presence or absence of case systems in languages. While \cite{croft2002typology} acknowledges the importance of case systems in certain contexts, such as with the English possessive \textit{'s'}, this aspect is not consistently addressed, especially when discussing morphologically unmarked genitives.

Furthermore, the term ``linker" used by \cite{croft2002typology} to refer to invariant elements on heads or dependents, or elements problematic for classification, lacks clarity and precision. This ambiguity hinders the accurate classification and analysis of possessive marking strategies across languages.

The shortcomings of \citeauthor{croft2002typology}'s \citeyear{croft2002typology} approach are considerable. While it introduces important observations, it also faces significant limitations. One key issue is the problematic nature of the classifier strategy within the classification framework proposed by \cite{nichols_locus_2013}. It is unclear whether a classifier should be associated with the head or the dependent, complicating the classification and analysis of possessive constructions.


\subsection{Summary}

Typology presented in \cite{nichols_locus_2013} classifies languages based on the locus of marking within possessive constructions, identifying head-marking, dependent-marking, and double-marking languages. However, this approach has limitations. It excludes constructions without overt possessors, limiting comprehensiveness and overlooking significant marking patterns. \cite{nichols_locus_2013} also unnecessarily distinguishes between morphologically marked and unmarked possessors, complicating classification without clear benefits. Furthermore, her simplification often conflates head-marking with double-marking strategies, as seen in languages like Mongolian where possessive markers are integrated affixes.

Croft's typology, described in \cite{croft2002typology}, aims to illustrate different marking patterns across languages rather than cover all linguistic variance. This approach, however, has notable shortcomings. Croft does not explicitly address the locus of marking, leading to classifications based on unspecified characteristics. He also overlooks variation within possessive constructions, particularly double-marked ones, and does not consistently consider the presence of case systems in languages. Croft's use of the term ``linker" for invariant elements is imprecise, complicating accurate classification.

To address these issues, I propose an inclusive typology that considers constructions without overt possessors and treats marked and unmarked possessors uniformly by recognizing zero marking in languages with grammaticalized case systems in the next chapter. Adopting a bottom-up strategy allows for more flexible and nuanced classification by categorizing markers as C-marking or D-marking, as suggested by \cite{lander2020head}. This method avoids classifying entire languages or constructions and focuses on individual markers. Clarifying the definition and role of linkers would also enhance the accuracy of Croft's typology. These adjustments aim to develop a more comprehensive framework for analyzing possessive marking across languages.

\section{Newer Approach}

As previously discussed, I propose classifying markers and their combinations instead of classifying constructions or languages as a whole. This approach avoids missing important relationships between markers that occur in different languages.

For instance, consider the languages Even (Tungusic), Kildin Saami (Saamic < Uralic), and Marind (Marind-Yaqay < Anim). Traditional classification would label all three as double-marking, but their behaviors differ significantly.

In Even, possessive suffixes and the genitive form of pronouns interact as follows:

\begin{enumerate}
	\item Juxtaposition is impossible.
	\item Using possessive suffixes with an unmarked possessor is impossible.
	\item Using possessive suffixes without an explicit possessor is possible.
	\item Using a marked possessor without possessive suffixes is impossible.
	\item Using a marked possessor alongside with a possessive suffix is possible.
\end{enumerate}

\pex
\glpreamble Even//
\a
\begingl
\gla\ljudge{*}bi ǯu(-wu) //
\glb I house-\Poss.\Fsg{} //
\glft `my house' //
\endgl

\a
\begingl
\gla min ǯu*(-wu) //
\glb I.\Obl{} house-\Poss.\Fsg{} //
\glft `my house' //
\endgl
\xe

In Even, head marking in possessive constructions is mandatory and cannot be omitted.

Marind, however, follows a different pattern. Its most common strategy involves the postposition \textit{en}, associated with the dependent. Head marking is present for many kinship terms, which have special forms marked with possessive prefixes. These special forms can be used alongside the \textit{en} construction, but it is not obligatory.

\begin{enumerate}
	\item Juxtaposition is possible.
	\item Using possessive prefixes with an unmarked possessor is possible.
	\item Using possessive prefixes without an explicit possessor is possible.
	\item Using a marked possessor without possessive prefixes is possible.
	\item Using a marked possessor alongside with a possessive prefix is possible.
\end{enumerate}

Another notable difference from Even is that the use of possessive suffixes in Marind is restricted to kinship terms. However, this difference likely stems from semantic restrictions, which are beyond the scope of this study. For my research, I will include any construction, regardless of semantic restrictions.

It is important to recognize that different interactions between markers can be distributed across the lexicon. For example, in Marind, juxtaposition is almost impossible for kinship terms but is commonly used with body parts. To make generalizations at such a fine-grained level, further research is required. In this study, I focus on the possibilities and impossibilities of these interactions.

Kildin Saami presents another case. Head-marking strategy in this language is obsolete, however occuring sporadically. The speakers allow using possessive suffixed on kinship terms, and some of them allow on animals. However, genitive marked possessors are quite common. Unlike Marind, Kildin Saami does not allow juxtaposition to express possessive realations. Thus, it behave as follows.

\begin{enumerate}
	\item Juxtaposition is impossible.
	\item Using possessive suffixes with an unmarked possessor is impossible.
	\item Using possessive suffixes without an explicit possessor is possible.
	\item Using a marked possessor without possessive suffixes is possible.
	\item Using a marked possessor alongside with a possessive suffix is possible.
\end{enumerate}

\subsection*{Summary}

Following my approach, systems of Even, Kildin Saami and Marind could be represented as follows. Tables below provide a detailed overview of how possessive markers interact in Even, Kildin Saami, and Marind.

\begin{table}[h!]
	\centering
	\begin{tabular}{@{}lll@{}}
		\toprule
		& D-marker & C-marker \\ \midrule
		Even         & 1        & 1        \\
		Kildin Saami & 1        & 1        \\
		Marind       & 1        & 1        \\ \bottomrule
	\end{tabular}
	\caption{Presence of D and C markers in Even, Kildin Saami and Marind}
\end{table}

% Please add the following required packages to your document preamble:
% \usepackage{booktabs}
\begin{table}[h!]
	\centering
	\small
	\begin{tabular}{@{}lccccc@{}}
		\toprule
		& juxtaposition & \parbox{2.5cm}{C-marker without possessor} &  \parbox{2.75cm}{C-marker with unmarked possessor} & \parbox{1.75cm}{C-marker \& D-marker} &  \parbox{3cm}{D-marker with unmarked possessum} \\ \midrule
		Even         & -             & +                          & -                                & +                    & -                                \\
		Kildin Saami & -             & +                          & -                                & +                    & +                                \\
		Marind       & +             & +                          & +                                & +                    & +                                \\ \bottomrule
	\end{tabular}
	\caption{Possible interactions between possessive markers in Even, Kildin Saami and Marind}\label{tab:even_saami_marind}
\end{table}

Every cell in Table \ref{tab:even_saami_marind} can be populated with specific information about the restrictions associated with each combination of markers. 

While I acknowledge the limitations of using a table format to represent the typology I propose, it remains the most convenient visualization method for languages like Even, Kildin Saami, and Marind. These languages each have only one D-marker and a single set of C-markers, making a table format suitable for describing the combinations of markers they exhibit. However, it's important to recognize that this format may not fully capture the complexity of marker interactions in languages with more diverse marker systems. Nevertheless, for the purposes of this study, the table provides a clear and structured overview of the restrictions associated with each combination of markers in these languages.

\section{Data Collection}

To comprehend the nature and potential diversity of possessive constructions, particularly in the context of head and dependent marking interaction, constructing a variety sample is essential. The aim of a variety sample is to capture a broad spectrum of linguistic patterns with minimal effort, as outlined by \cite{miestamo2016sampling} following \citealt{rijkhoff1993method}.

Another approach to sampling is probability sampling, which allows for statistical generalizations based on language data. However, as discussed in the literature on language sampling, achieving a completely unrelated or independent sample is practically unattainable. In any large language sample, there are bound to be relationships among languages, whether they are genetic, areal, cultural, or otherwise interconnected.

\subsection{Probability Sampling and Concept of Genus}

To understand if there is a tendency to express something in a particular way, it is impractical to examine all languages of the world for several reasons. First, it is practically impossible given the large number of languages. Second, such a comprehensive survey would not necessarily yield accurate conclusions. \cite{dryer1989large} illustrated this point through a thought experiment. Imagine a hypothetical world consisting of 1000 languages, categorized as follows:

\begin{itemize}
	\item 900 languages originate from a single language family.
	\item The remaining 100 languages originate from ten different families.
\end{itemize}

Now, suppose in this world, the majority of languages (900 out of 1000 originating from a single family) exhibit SVO word order, while the remaining 100 languages demonstrate SOV word order. Despite observing a prevalence of SVO word order, we cannot conclusively state that there is an overall tendency towards SVO word order across languages. This observation is likely influenced by genealogical bias, where the high occurrence of SVO word order in the majority of languages can be attributed to their shared linguistic ancestry within the same family. Therefore, drawing generalizations about linguistic tendencies requires careful consideration of language relationships to avoid misleading interpretations based solely on observed patterns in a diverse linguistic landscape.


\citet{dryer1989large} and \citet{bickel2008refined}, among others, proposed methods for creating a probability sample in linguistic studies. \citet{dryer1989large} introduced the concept of ``genus'' to mitigate genealogical bias. A genus refers to a grouping of languages with a time depth not exceeding 3,500 to 4,000 years \citep{dryer1989large}. These genera are considered sufficiently independent to be included in the sample. Another criterion for independence is the distant geographic location of languages. \citet{dryer1989large} argues that so-called Macro-Areas (Africa, Eurasia, Australia + New Guinea, North America, South America) are independent from each other in terms of linguistic properties. The main idea is to assign values to all genera (allowing for multiple values since languages within a genus can exhibit different behaviors) and then create a table similar to Table \ref{dryer_table}.

% Please add the following required packages to your document preamble:
% \usepackage{booktabs}
\begin{table}[ht]
	\centering
	\begin{tabular}{@{}ccccccc@{}}
		\toprule
		\multicolumn{1}{l}{} & \textbf{Africa} & \textbf{Eurasia} & \textbf{Australia-NG} & \textbf{North America} & \textbf{South America} & \textbf{Total} \\ \midrule
		\textbf{SOV}         & 22              & 26               & 19                    & 26                     & 18                     & 111            \\
		\textbf{SVO}         & 21              & 19               & 6                     & 6                      & 5                      & 57             \\ \bottomrule
	\end{tabular}
	\caption{Word order in different genera across Macroareas}\label{dryer_table}
\end{table}

Next, \citet{dryer1989large} determines the prevalence of SOV or SVO word order within each Macroarea. Modeling this scenario as a binomial distribution, he suggests that there is a tendency towards SOV, reasoning that the probability of SVO ``winning'' five times (with a probability of 0.5 for each outcome) is less than 0.05 ($0.5^5 = 0.03125$), indicating a statistically significant preference for SOV word order.

While the methodology involving genera and macroareas may raise questions, the concepts have been employed in other studies dedicated to language sampling. \citet{miestamo2016sampling} proposed a method for creating a variety sample based on genera and macroareas, following the framework established by \citet{dryer1989large} regarding the relative independence of genera and macroareas. In variety sampling, complete independence is not mandatory. \citet{miestamo2016sampling} divided the world into six macroareas, with a distinct categorization for Australia and New Guinea. The Genus-Macroarea method involves selecting languages from a macroarea in proportion to the number of genera within that macroarea, aiming to achieve a representative and diverse sample of languages across different geographic and genetic classifications.

\subsection{Automatic Sampling}

Since genera are defined on the genealogical basis, it is important to be competent enough for doing it. For now, the only study where genera are systematically annotated is WALS \citep{wals}. The main problem of WALS is that it is a frozen project and no new languages could be added and consequently receive a genus.

WALS contains 2,662 languages while Glottolog \citep{hammarstrom2014glottolog} contains 8,604 languages. Thus even if Glottolog is reducable to WALS (which it is not), the maximum coverage of genus annotation would be no more than 30 percent of all languages. This is an important methodological problem. While one can use sampling methods involving genera, they are limited to least than a half languages of the world.

\subsubsection{Automatic Annotation of Genera}

In the course of this thesis, one of my objectives is to enhance the database of languages for which genera are identified. It is worth noting that whether or not one subscribes to the concept of genera from a theoretical standpoint is not crucial. Genera serve as practical tools for assembling a diverse sample.

The approach to extend the number of known genera is straightforward. If we establish that certain languages are classified under the same genus in WALS, then it follows that all languages stemming from their earliest shared ancestor fall within this specific genus. This method provides a simple yet effective means of augmenting our understanding of language diversity.

Initially, it was imperative to transform the data from Glottolog into a machine-readable format while preserving the genealogical information. Subsequently, this data needed to be converted into a tree structure using Python. The next step involved identifying the shared ancestor and assigning all of its descendants to the appropriate genus.

However, this process posed several potential challenges. Firstly, sign languages and creoles needed to be excluded. Although WALS categorizes them under specific genera, Glottolog distributes them differently. Secondly, discrepancies between WALS and Glottolog were encountered. Sometimes, following the algorithm described above resulted in conflicts.

For instance, WALS classifies Khanty, Mansi, and Hungarian under the Ugric genus. However, recent research has demonstrated that this genealogical grouping is not accurate, as common innovations are attributed to extensive language contact instead. Glottolog adopts a newer classification (Figure \ref{fig:new_uralic}), whereas WALS utilizes the one depicted in Figure \ref{fig:old_uralic}. Consequently, applying the algorithm would erroneously assign the Ugric genus to all Uralic languages.

\begin{figure}
	\dirtree{%
		.1 Uralic.
		.2 Finnic.
		.2 Hungaric.
		.3 Hungarian.
		.3 Old Hungarian.
		.2 Khantyic.
		.3 East Khanty.
		.4 Surgut Khanty.
		.4 Vach-Vasjugan.
		.3 Northern Khanty.
		.4 Kazym-Berezover-Suryskarer Khanty.
		.4 Obdorsk Khanty.
		.3 Sothern Khanty.
		.2 Mansic.
		.3 North-Central Mansi.
		.4 Central Mansi.
		.4 Northern Mansi.
		.3 Southern Mansi.
		.2 Mari.
		.2 Mordvin.
		.2 Permian.
		.2 Saami.
		.2 Samoyedic.
	}
	\caption{Uralic languages according to Glottolog}\label{fig:new_uralic}
\end{figure}

\begin{figure}
	\dirtree{%
		.1 Uralic.
		.2 Ugric.
		.3 Khanty.
		.3 Mansi.
		.3 Hungurian.
	}
	\caption{Ugric languages according to WALS}\label{fig:old_uralic}
\end{figure}

Given the impossibility of achieving a perfect match between WALS and Glottolog, the decision was made to leave all nodes unchanged if any of their descendants were assigned a different genus by WALS initially.

The proposed method operates independently of genealogical expertise, relying instead on logical principles to yield accurate results, barring errors present in the databases.

Following the initial merging of WALS with Glottolog, only 28.1 percent of languages in Glottolog were annotated with a genus. However, after implementing the algorithm and accounting for isolates not included in WALS, the coverage increased significantly to 75.8 percent. In practical terms, this means that I successfully annotated 6,506 languages from Glottolog with their respective genera.

\subsubsection{Making Sample}

I utilized the Genus-Macroarea sampling method proposed by \cite{miestamo2016sampling} and automated by \cite{cheveleva2023}. This method involves selecting a proportional number of languages from each of the six macroareas based on the total number of genera within each macroarea. While \cite{cheveleva2023} recalculated the proportions for each macroarea using a newer version of WALS, I performed a further recalculation to ensure the most accurate data. The recalculated proportions are detailed in Table \ref{tab:proportions}.

% Please add the following required packages to your document preamble:
% \usepackage{booktabs}
\begin{table}[h]
	\centering
	\begin{tabular}{@{}lll@{}}
		\toprule
		& Glottolog & \%   \\ \midrule
		Africa        & 131       & 18.1 \\
		Australia     & 44        & 6    \\
		Papunesia     & 200       & 27.7 \\
		Eurasia       & 96        & 13.3 \\
		North America & 112       & 15.5 \\
		South America & 139       & 19.3 \\
		Total         & 722       & 100  \\ \bottomrule
	\end{tabular}
	\caption{Genera and languages by macroarea}\label{tab:proportions}
\end{table}


Inspired by the methodology proposed by \cite{cheveleva2023}, I undertook the task of annotating the number of pages for each grammar available for every language. Unlike the previous study, I utilized more comprehensive data and accounted for automatically identified sources. Notably, Glottolog employs an internal algorithm to automatically assess the classification of new items in the bibliography. Items automatically annotated as grammars were excluded from the database compiled by \cite{cheveleva2023} due to their distinct tag.

Moreover, I made modifications to the automatic sample creation algorithm originally proposed by \cite{cheveleva2023}. While retaining the concept of automatic suggestion of genera, I introduced a more manual selection process. My algorithm now suggests genera that warrant closer examination, allowing linguists to make informed decisions regarding language selection. To facilitate this process, I created individual tables for each genus, providing information on the lengths of available grammars and the rationale for assigning each language to a specific genus. In cases where a language was assigned to a genus based on a common ancestor with two other languages, I included details about the common ancestor and the related languages.

Additionally, I implemented lists of prechosen and banned genera to further refine the sampling process. The prechosen genera list enables linguists to select genera based on their personal familiarity or the availability of data. Conversely, the banned genera list allows linguists to exclude genera for which they could not find suitable grammatical descriptions, thereby necessitating a re-sampling of languages.

These modifications have facilitated a more nuanced sampling process, affording greater control over each step of language selection and enhancing the overall quality of the sample.

\subsubsection{Sample}

Resulting sample includes the following genera and languages:

\begin{itemize}
	\item \textbf{Papunesia}: Marind (Marind-Yaqay), To'abaita (Oceanic), Lundayeh (North Borneo), Kobon (Kalam-Kobon), West Coast Bajau (Sama-Bajaw), Iloko (Northern Luzon)
	\item \textbf{South America}: Chácobo (Panoan), Hup (Nadahup), Yucuna (Japura-Colombia), Kwaza (isolate)
	\item \textbf{Eurasia}: Even (Tungusic), Kildin Saami (Saami), Abaza (Northwest Caucasian), Russian (Slavic)
	\item \textbf{North America}: Poqomam (Mayan), Central Alaskan Yupik (Eskimo), Haida (Haida)
	\item \textbf{Africa}: Ewe (Gbe), Ruund (Bantu), Paku (Barito), Lamang (Biu-Mandara)
	\item \textbf{Australia}: Ngardi (Western Pama-Nyungan)
\end{itemize}

It is imbalanced towards Eurasia since I have a field data for all Eurasia languages presented in the sample.




%\section{Other questions}
%
%- How do possessive markers undergo grammaticalization? (Нужно почитать джоанну)
%
%- Regarding the relative order of case and possessive markers: Dékány?
%
%- Is there a relationship between the markedness of the nominative form and the relative order of case and possessive markers?
%
%- Do marking strategies at the clause level, especially considering pro drop, correlate with strategies within NP? Verification of Nichols' observations.
%
%- Do markers on the dependent reflect registration while markers on the head reflect indexing (or do they exhibit no preference)? What does this imply about the "universal" structure of NP?
%
%- Are there any significant asymmetries between verbal agreement and nominal agreement? I'm aware of one epiphenomenal difference. It's difficult to find instances where Agreement failure elicits default agreement, as often occurs with sentential subjects in many languages. It's challenging to imagine a scenario where a non-nominalized clause could occupy the possessor's position.

\section{Possessive Systems}

\label{chapter:possessive_systems}

\subsection{Bystraja Even}

In Even, nominal possessive markers can be categorized into two subsets: personal and reflexive. Personal possessive markers indicate specific person values, including first (inclusive and exclusive in plural), second, and third persons. Reflexive possessive markers, as discussed by Buzanov (2022), are employed when there is co-reference between the possessor of a noun and the subject of the entire clause. While reflexive possessive markers are used with possessors of any person value, they do not differentiate between different person values. Both personal and reflexive markers distinguish between two number values: singular and plural. The paradigm for these markers is illustrated in Table \ref{even_poss}.

\begin{table}[ht]
	\caption{Possessive markers in Even}\label{even_poss}
	\begin{subtable}[t]{0.48\textwidth}
		\centering
		\caption{\centering Personal possessive suffixes}
		\label{perposs}
		\begin{tabular}[t]{ll}\addlinespace\toprule
			\Poss    & morpheme  \\\midrule
			\Fsg     & -wu   \\
			\Ssg     & -š(i)     \\
			\Tsg     & -n(i)       \\
			\Fpl.\Incl & -t(i)     \\
			\Fpl.\Excl & -wun \\
			\Spl     & -šan      \\
			\Tpl     & -tan      \\\bottomrule
		\end{tabular}
	\end{subtable}
	\hfill
	\begin{subtable}[t]{0.48\textwidth}
		\centering
		\caption{\centering Reflexive possessive suffixes}
		\label{reflposs}
		\begin{tabular}{cl}\addlinespace\toprule
			\Poss.\Refl & morpheme\\\midrule
			\Sg & -i/-ji/-mi/-bi \\
			\Pl & -wur\\\bottomrule
		\end{tabular}
	\end{subtable}
\end{table}

Another way of expressing possessive relations in Even is through the possessive form of a (pro)noun. Both personal pronouns and the reflexive pronoun in Even have distinct possessive forms, which I will refer to as the genitive form. These forms can be utilized within possessive constructions alongside the corresponding possessive suffixes detailed in Table \ref{even_poss}, or within postpositional phrases, which should be analyzed as possessive phrases.

The genitive stem is not only used for possessive forms but also to derive other case forms of pronouns, as illustrated in Table \ref{even_pronouns}. Unlike pronouns, nouns do not have special genitive forms. Nouns in the possessor position remain morphologically unmarked, much like their appearance in subject positions. Third-person pronouns exhibit noun-like behavior, evident in their morphological structure, which includes distinct non-suppletive exponents for number and possessive marking.

The first-person inclusive pronoun \textit{mut} also displays similar forms in both subject and possessor positions, which is unexpected.

\begin{table}[t]
	\centering
	\begin{tabular}[t]{llll}\addlinespace\toprule
		pronoun    & \Nom & \Gen & \Dat  \\\midrule
		\Fsg     & bi & min &  min-du  \\
		\Ssg     & i & in & in-du    \\
		\Tsg     & noŋa-n `(s)he-\Poss.\Tsg' & noŋa-n & noŋan-du-n `(s)he-\Dat-\Poss.\Tsg' \\
		\Fpl.\Incl & mut & mut & mut-tu \\
		\Fpl.\Excl & bu & mun & mun-du \\
		\Spl     & u & un & un-du     \\
		\Tpl     & noŋa-r-tan `(s)he-\Pl-\Poss.\Tpl' & noŋa-r-tan  & noŋa-r-du-tan `(s)he-\Pl-\Dat-\Poss.\Tpl'\\\bottomrule
	\end{tabular}
	\caption{\centering Personal pronouns in Even}
	\label{even_pronouns}
\end{table}

It is important to note that while possessive suffixes are obligatory, these genitive forms are optional (see \ref{even_optional}).

\ex
\label{even_optional}
\begingl
\gla min ǯu-wu//
\glb I.\Obl{} house-\Poss{}.\Fsg//
\glft `my house'//
\endgl
\xe 

In Even, compounds are absent. Instead, compound-like structures are conveyed through possessive phrases, exemplified in \ref{even_berry}.

\ex
\begingl
\gla munnukan təwtə-n//
\glb hare berry-\Poss.\Tsg//
\glft `cranberry' (lit. `hare\'s berry')//
\endgl
\xe

\subsubsection*{Summary on Even}

In Even, there are markers that appear on the head of possessive phrases, along with the genitive form of (pro)nouns, which is an instance of dependent marking.

While dependent marking is optional, head marking is obligatory and resemble agreement, occurring even with non-referential possessors.

\subsection{Kildin Saami}

According to the possessive paradigms presented in \cite{kuruch_saamsko-russkij_1985} and \cite{kert_saamskij_1971}, in Kildin Saami, there are three distinct possible person values as illustrated in Tables \ref{saami_poss1} and \ref{saami_poss2}. These scholars do not mention any variance between speakers. However, \cite{riesler_kildin_2022} notes that ``Kildin Saami has lost the regular possessive inflection of nouns. Remnants of the former possessive inflection are only found occasionally with kinship nouns".

\begin{table}[ht]
	\centering
	\begin{tabular}{ccccccl}
		\toprule
		\multicolumn{1}{c}{\multirow{2}{*}{Case}} & \multicolumn{3}{c}{\Sg{} possessor, \Sg{} possessum} & \multicolumn{3}{c}{\Pl{} possessor, \Sg{} possessum}\\
		\multicolumn{1}{c}{} & 1 person & 2 person & 3 person & 1 person & 2 person & 3 person \\\midrule
		
		Nominative & \parbox{1.5cm}{-a{[}m{]} \\ -аn} & -at & -es' & -a{[}m{]} / -аn & -ant & -edes' \\\addlinespace
		
		Genitive & -аn & -at & -es' & -an / -edan & -еdant & -edes' \\\addlinespace
		
		Accusative & -аn & -at & -es' & -e(t)dan & -еdant & -edes' \\\addlinespace
		
		Essive & -jan & -jant & -jas ' & -jedan & -jedant & -jedes' \\\addlinespace
		\parbox{2cm}{Inessive-Elative} & -san & -sant & -esan & -esan & -esant & -eses' \\\addlinespace
		
		\parbox{2cm}{Dative-Illative} & -(ja)san & -(je)sant & -jes' & -jedan & -jedant & -jedas' \\\addlinespace
		
		\bottomrule
	\end{tabular}
	\caption{Part of Possessive Paradigm of Kildin Saami}\label{saami_poss1}
\end{table}

\begin{table}[ht]
	\centering
	\begin{tabular}{ccccccl}
		\toprule
		\multicolumn{1}{c}{\multirow{2}{*}{Case}} & \multicolumn{3}{c}{\Sg{} possessor, \Pl{} possessum} & \multicolumn{3}{c}{\Pl{} possessor, \Pl{} possessum}\\
		\multicolumn{1}{c}{} & 1 person & 2 person & 3 person & 1 person & 2 person & 3 person \\\midrule
		
		Nominative & \parbox{1.5cm}{-a{[}m{]} \\ -аn} & -ant & -edes' & -edan & -edant & -edes' \\\addlinespace
		
		Genitive & -edаn & -edant & -edes' & -edan & -еdant & -edes' \\\addlinespace
		
		Accusative & -edаn & -edant & -edes' & -edan & -еdant & -edes' \\\addlinespace
		
		Essive & -edаn & -jedant & -jedes' & -jedan & -jеdant & -jedes' \\\addlinespace
		\parbox{2cm}{Inessive-Elative} & -esan & -esant & -eses't' & -esan & -esan & -eses't' \\\addlinespace
		
		\parbox{2cm}{Dative-Illative} & -ejdan & -jedant & -jedas & -jedan & -jedant & -jedas \\\addlinespace
		\bottomrule
	\end{tabular}
	\caption{Part of Possessive Paradigm of Kildin Saami}\label{saami_poss2}
\end{table}

In my view, the assertion made by \cite{riesler_kildin_2022} is not entirely accurate. While it is true that the possessive declension is nearly extinct for nouns (as is the case in other Saami languages), it had been grammaticalized in reflexives and reciprocals prior to its loss. Therefore, possessive markers are obligatory in these structures, as demonstrated in example (\ref{refl_recip}).

\pex \label{refl_recip}
\a \begingl
\gla nɨzan ujjn-av kaan'n'c' kaan'n'c'-*(es')//
\glb women see-\Npst.\Tpl{} friend friend-\Poss{}\Third{}//
\glft `The women see each other.'//
\endgl
\a \begingl
\gla par̥'r̥'š'a oaffk iž'-*(es')//
\glb boy scolds self-\Poss{}\Third{}//
\glft `The boy scolds himself.'//
\endgl
\xe

Given that reflexives and reciprocals cannot be formed without the use of possessive morphology, yet possessive morphology is not commonly employed in the language, it is understandable that this presents an intriguing area of study.

Conventional possessive forms are infrequent and only appear sporadically. Nevertheless, these sporadic instances align with the diversity found in reflexive pronoun formation.

Speakers of Kildin Saami may be classified into two distinct groups based on the number of person values they differentiate in the possessive declension. One group of speakers differentiates between all three values, while the other group distinguishes only between two. In the latter group, first and second person values are always combined and expressed using a single marker, which consistently appears as \textit{-ant}, the ex-marker for second person. This differentiation is illustrated in (\ref{two_types}), with different colors representing each group of speakers.

\pex\label{two_types}
\a\begingl
\gla munn iž'-\textcolor{orange}{an}/-\textcolor{purple}{ant} šobbš-a//
\glb I self-\textcolor{orange}{\Poss{}\First}/\textcolor{purple}{\Poss{}\First\_\Second{}} love-\Npst{}\Fsg//
\glft `I like myself.'//
\endgl

\a\begingl
\gla toonn	soagg-ex	kul'	iiǯ-s'-\textcolor{orange}{ant}/\textcolor{purple}{ant}//
\glb you.\Sg{} catch-\Pst.\Ssg{} fish.\Acc{} self-\textcolor{orange}{\Poss{}\Second{}}/\textcolor{purple}{\Poss{}\First\_\Second{}}//
\glft `Did you fish for yourself?'//
\endgl

\xe

In Kildin Saami, speakers have different preferences regarding which nouns can take possessive suffixes. However, they generally adhere to a hierarchy: kinship terms > domestic animals > certain household items. This hierarchy implies that all speakers allow possessive markers on (some) kinship terms. If a speaker allows possessive marking on certain household items, they typically also allow possessive markers on certain domestic animals. Nouns that do not fit into these categories are less likely to be found with possessive markers.

In Kildin Saami, there is a set of possessive markers that are dependent-marked, known as possessive pronouns. These pronouns have largely replaced possessive suffixes in usage. Speakers of Kildin Saami prefer using these possessive pronouns, with possessive suffixes being rarely employed, except in cases involving reflexives and reciprocals as mentioned earlier.

\subsubsection*{Summary on Kildin Saami}

Kildin Saami has the preference for dependent marked possessive markers, specifically possessive pronouns, over possessive suffixes. These possessive pronouns have largely replace the use of possessive suffixes. Speakers of Kildin Saami commonly use possessive pronouns to indicate possession, with possessive suffixes reserved for specific contexts such as reflexives and reciprocals.

Furthermore, Kildin Saami speakers exhibit a hierarchical pattern in determining which nouns can take possessive markers. This hierarchy prioritizes certain categories of nouns, such as kinship terms, domestic animals, and household items, for possessive marking. Nouns falling outside these categories are less likely to be marked with possessive suffixes or pronouns, and when they are, it occurs sporadically. Concluding, possessive suffixes in Kildin Saami are not obligatory while possessive pronouns are obligatory.

\subsection{Marind}

In Marind, possessive relation can be expressed by two different strategies: a postpositional phrase headed by \textit{en} expressing the possessor (\ref{marind_en}), and juxtaposition of the possessor and possessum (\ref{marind_just}).

According to \citep[158]{olsson2021grammar}, \textit{en} is dependent marking since it forms a constituent with a possessum.

\ex
\label{marind_en}
\begingl
\gla amay en yay en pula//
\glb ancestor \Poss{} uncle \Poss{} taboo.spot//
\glft `grandpa\'s uncle\'s taboo spot'//
\endgl
\xe

\ex
\label{marind_just}
\begingl
\gla nok onos ɣa k-a-Ø ehe, oɣ ɣakna k-a-Ø//
\glb 1 cousin real aprs.ntrl-3sg-be.npst prox:I 2sg husband's.elder.bro:2sg aprs.ntrl-3sg-be.npst//
\glft `This is my cousin, your brother-in-law.'//
\endgl
\xe

According to \cite{olsson2021grammar}, the postpositional phrase with \textit{en} in Marind has no restrictions on the type of ownership expressed and can convey associative meanings. This construction is flexible in its use and allows for a broad range of possessive relationships to be expressed.

In Marind, many kinship terms have special forms marked with possessive prefixes that are identical to the Undergoer prefixes. While these special marked forms can be used along with the \textit{en} construction, it is not obligatory to do so. Using unmarked possessor is sufficient to express possessive relation.


Notably, the juxtaposition strategy of placing the possessor and possessum side by side is not commonly observed with kinship terms in Marind, highlighting a preference for the \textit{en} postpositional phrase when expressing possession involving kinship terms.

\subsubsection*{Summary on Marind}

In Marind, possessive marking involves two distinct sets of markers. The first set utilizes dependent marking with the postposition \textit{en}, which establishes a relationship between the possessor and the possessum within a postpositional phrase. This strategy is versatile and can convey various types of possession and associative meanings.

The second set of markers consists of prefixes occurring on the head of most kinship terms. These special prefixes indicate possession and are closely associated with kinship relationships. The presence of a special prefixed form for a noun implies that the \textit{en} strategy can also be used for possession involving that noun.

However, if there is no special prefixed form available for a noun, speakers of Marind have the option to use either the \textit{en} strategy or simply place the possessor and possessum side by side without any intervening markers. Juxtaposition, in this context, represents a no-marking strategy since Marind lacks case marking.

\subsection{Chacobo}

In Chacobo, possessive relations are expressed primarily through the use of genitive noun phrases and possessive pronouns. Genitive noun phrases are dependents in the NP-constituent that precede the head noun. These possessive NP are marked with a high tone clitic =\textasciiacute{} `genitive' (\ref{chacobo_genitive}), and they precede noun phrase dependents in noun compounds. 

\ex
\label{chacobo_genitive}
\begingl
\gla noʔó haʔɨpa//
\glb 1sg father//
\glft `my father'//
\endgl
\xe

Two morphemes that denote ``father'' \textit{-ipa} and ``mother'' \textit{-ɨpa} cannot occur without some formal marking of possession. The possession is marked with person-number prefixes such as \textit{mi-} for second person and \textit{ha-} for first/third person. These prefixes must accompany the nouns to denote possession. These possessed forms can be combined with possessive pronouns as well.

The juxtaposition of possessor and possessum without any intervening markers is not a common strategy in Chacobo, emphasizing a preference for clear possessive marking either through clitics or prefixes.

\begin{table}[h!]
	\centering
	\small
	\begin{tabular}{@{}cccccc@{}}
		\toprule
		& juxtaposition & \parbox{2.5cm}{C-marker without possessor} & \parbox{2.75cm}{C-marker with unmarked possessor} & \parbox{1.75cm}{C-marker \& D-marker} & \parbox{3cm}{D-marker with unmarked possessum} \\ \midrule
		Chacobo & - & + & - & + & + \\ \bottomrule
	\end{tabular}
	\caption{Possessive marking strategies in Chacobo}
\end{table}

\subsection{Ewe}

Ewe possessive constructions can be expressed through several methods:

\begin{itemize}
	\item \textbf{Possessive Linker}: This method involves a possessive linker \textit{fe}, and is used in the `alienable nominal construction' where the structure is NP$_\textsc{possessor}$ fe NP$_\textsc{possessum}$ (\ref{ewe_example}).
	
	\item \textbf{Juxtaposition}: This method, known as the `inalienable nominal construction', also follows the structure NP$_{possessor}$ NP$_\textsc{possessum}$ without any connective.
	
	\item \textbf{Syntactic Compounding}: In this structure, the two nominals are compounded and marked with a high tone suffix at the end: N$_\textsc{possessor}$- N$_\textsc{possessum}$ + HIGH TONE SUFFIX.
\end{itemize}


\ex
\label{ewe_example}
\begingl
\gla kofi fe awu vu//
\glb K. \Poss{} dress tear//
\glft `Kofi's garment is torn.'//
\endgl
\xe

If the possessor is realised as a pronoun, a variant of the independent forms are used. Apart from the first and second person singular forms, all other pronouns are linked to the possessed items by the possessive connective (cf. \ref{ewe_pronoun_example} and \ref{ewe_pronoun_example2}).

\ex
\label{ewe_pronoun_example}
\begingl
\gla mia fe agble- xo kió//
\glb 1\textsc{pl} \Poss{} farm house bare//
\glft `Our farm house is without a roof.'//
\endgl
\xe

\ex
\label{ewe_pronoun_example2}
\begingl
\gla nye (*fe) ga bú//
\glb 1\textsc{sg:poss} \Poss{} money lost//
\glft `My money is lost.'//
\endgl
\xe


\begin{table}[h!]
	\centering
	\small
	\begin{tabular}{@{}cccccc@{}}
		\toprule
		& juxtaposition & \parbox{2.5cm}{C-marker without possessor} & \parbox{2.75cm}{C-marker with unmarked possessor} & \parbox{1.75cm}{C-marker \& D-marker} & \parbox{3cm}{D-marker with unmarked possessum} \\ \midrule
		Ewe & + & - & + & - & + \\ \bottomrule
	\end{tabular}
	\caption{Possessive marking strategies in Ewe}
\end{table}


\subsection{Ruund}

Nouns in Bantu languages are commonly modified using associative phrases, known as the ``associative construction''. In this construction, the head noun is followed by an associative phrase that includes an associative particle (linker) and a modifying noun. The associative particle, which conveys a meaning similar to ``of'', connects the modifying noun to the head noun, indicating possession or association.

The associative particle is a combination of the pronominal concord for the head noun's class and the suffix /-a/. From a phonological perspective, associative particles function more like prefixes than separate words. They resyllabify with a following nasal prefix, creating a long vowel through the Vowel Lengthening rule. Due to this fact, I consider this as a dependent marking strategy.

In Ruund, there is another construction that utilizes an associative particle to indicate possession. This construction involves the following elements: a possessum, an associative particle that agrees with both the possessum (prefix slot) and the possessor (suffix slot), and the possessor itself (\ref{ex:cringe_ruund}).

\ex\label{ex:cringe_ruund}
\begingl
\gla yiis y-aa-c cikumbu//
\glb doors(8) \textsc{c8}-\Lnk-\textsc{c7} the.house(7)//
\glft `the house's doors'//
\endgl
\xe

No head marking elements to indicate possessive relations are present in Ruund. Juxtaposition is also impossible.

\begin{table}[h!]
	\centering
	\small
	\begin{tabular}{@{}cccccc@{}}
		\toprule
		& juxtaposition & \parbox{2.5cm}{C-marker without possessor} & \parbox{2.75cm}{C-marker with unmarked possessor} & \parbox{1.75cm}{C-marker \& D-marker} & \parbox{3cm}{D-marker with unmarked possessum} \\ \midrule
		Ruund & - & NA & NA & NA & + \\ \bottomrule
	\end{tabular}
	\caption{Possessive marking strategies in Ruund}
\end{table}

\subsection{Paku}

In Paku, possession of a noun can be indicated through two methods: nominal possession and singular pronominal possession. Nominal possession involves juxtaposing the possessor and the possessum, with no further modification required for either noun. An example of this is shown in (\ref{paku_nominal}), where the head noun \textit{tu'ulang} `bone' is immediately followed by the possessor \textit{eteng} `dog', resulting in a possessive interpretation of the phrase.

\ex
\label{paku_nominal}
\begingl
\gla tu'ulang eteng//
\glb bone dog//
\glft `the dog's bone'//
\endgl
\xe

Pronominal possession, on the other hand, depends on the number of possessors. If the possessor is singular, a possessive enclitic is attached to the nominal possessum. An example of this is shown in (\ref{paku_pronominal}), where the first person singular possessive pronoun \textit{-ku} follows the head noun \textit{adi'} `younger sibling'.

\ex
\label{paku_pronominal}
\begingl
\gla anrape adi'-ku N-wintan pito kV-ukui kenah //
\glb yesterday younger.sibling-\Fsg.\Poss{} \Av{}-fish seven kV-\Clf2 fish//
\glft `Yesterday my little brother caught seven fish.'//
\endgl
\xe

In cases where the possessor is plural, possession is expressed similarly to nominal possessive constructions, with the personal pronoun following the possessed noun.

\begin{table}[h!]
	\centering
	\small
	\begin{tabular}{@{}cccccc@{}}
		\toprule
		& juxtaposition & \parbox{2.5cm}{C-marker without possessor} & \parbox{2.75cm}{C-marker with unmarked possessor} & \parbox{1.75cm}{C-marker \& D-marker} & \parbox{3cm}{D-marker with unmarked possessum} \\ \midrule
		Paku & + & + & - & NA & NA \\ \bottomrule
	\end{tabular}
	\caption{Possessive marking strategies in Paku}
\end{table}


\subsection{Lamang}

The standard way of constructing associative phrases with nouns in Lamang involves the first noun carrying two markers specific to this construction, regardless of whether the possessor is represented by another noun or a personal pronoun (referred to as a ``possessive pronoun''). This construction follows a pattern of N-á-a N/poss, where the first noun takes on the attributive modifier {-a}. The combination of the attributive marker {-a} and the preceding vowel forms the final syllable, which bears stress. When the possessor is personal pronoun it attaches to the possessum as a suffix (\ref{ex:lamang}).

\pex\label{ex:lamang}
\a *agu-á-a´ dada [ó"gáa dàdà] `father's goat'
\a *agu-á-ha-a´ dada [ógá"háa dàdà] `father's goats'
\a *agu-á-a-ìnì [ó"gìinì] `his/her goat'
\xe


\begin{table}[h!]
	\centering
	\small
	\begin{tabular}{@{}cccccc@{}}
		\toprule
		& juxtaposition & \parbox{2.5cm}{C-marker without possessor} & \parbox{2.75cm}{C-marker with unmarked possessor} & \parbox{1.75cm}{C-marker \& D-marker} & \parbox{3cm}{D-marker with unmarked possessum} \\ \midrule
		Lamang & - & + & + & NA & NA \\ \bottomrule
	\end{tabular}
	\caption{Possessive marking strategies in Lamang}
\end{table}

\subsection{Ngardi}

When the possessor functions as the head of a noun phrase, various morphological strategies are employed to denote a possessive relationship. These include the use of the proprietive suffix, the dative suffix -\textit{ku}, the possessive suffix -\textit{kuny}, the personal possessive suffix -\textit{punta}, and two kinship suffixes: the anaphoric propositus -\textit{nyanu} and the possessed kin -\textit{nguniny}.

Internal possession constructions are characterized by all possession markers occurring within the noun phrase. If such a possessive construction occupies an argument position within the clause, the entire possessor phrase is cross-referenced. Bound pronouns are not utilized to indicate the possessor. Internal possession can be further classified into two subtypes based on whether the possessor or the possessum is marked for the possessive relationship.

In the first subtype, the possessum may be marked with the `anaphoric propositus' suffix -\textit{nyanu} or the `possessed kin' suffix -\textit{nguniny}, indicating kinship or close relations, as in (\ref{ex:nyanu}).

\ex \label{ex:nyanu}
\begingl
\gla Kuyi=pula=rlapp [ngati-nyanu-ku]pp ka-ngu-ngkarla yi-nya-ngurra.//
\glb meat=3du.s=3sg.obl M-anaph=3sg.obl carry-inf-seq.loc give-pst-narr//
\glft 'The two gave the meat to their mother, having carried it.'//
\endgl
\xe


Alternatively, the possessor may be marked while the possessum remains unmarked. Possessor-marking suffixes such as the possessive -\textit{kuny}, the personal possessive -\textit{punta}, or the dative -\textit{ku} are employed. Since adnominal case-marked noun phrases in Ngardi typically receive additional case markings indicating their role in the clause, the possessive -\textit{kuny} suffix is often utilized in marking internal possession, as in (\ref{poss_example}).

\ex \label{poss_example}
\begingl
\gla Paja-rni=lu ngaju-kuny-ju kunyarr-u.//
\glb bite-pst=3pl.s 1sg-poss-erg dog-erg//
\glft 'My dogs killed it.'//
\endgl
\xe


\begin{table}[h!]
	\centering
	\small
	\begin{tabular}{@{}cccccc@{}}
		\toprule
		& juxtaposition & \parbox{2.5cm}{C-marker without possessor} & \parbox{2.75cm}{C-marker with unmarked possessor} & \parbox{1.75cm}{C-marker \& D-marker} & \parbox{3cm}{D-marker with unmarked possessum} \\ \midrule
		Ngardi & - & + & - & - & + \\ \bottomrule
	\end{tabular}
	\caption{Possessive marking strategies in Ngardi}
\end{table}

\subsection{Mongolian}

When the possessor is omitted, possessive suffixes are used as in \parref{mongolc} where the possessor `he/she/they' seems to be phonologically null (\textit{pro}) but is somehow marked on the head noun \textit{egc} `elder sister' with the suffix \textit{-e.n'} \textsc{poss.3}. And the meaning is combined perfectly.

 \pex
\label{mongol2}
\glpreamble Mongolian (Mongolic) from \cite[138]{janhunen_mongolian_2012}//
\a
\label{mongola}
\begingl
\gla iljeg/n-ii cix-(*e.n')//
\glb donkey-gen ear-(*\Poss{}.\Third{})//
\glft `Donkey\'s ear.'//
\endgl

\a
\label{mongolb}
\begingl
\gla iljeg/n cix-(*e.n')//
\glb donkey ear-(*\Poss{}.\Third{})//
\glft `Date (fruit).'//
\endgl

\a
\label{mongolc}
\begingl
\gla egc-e.n'//
\glb elder.sister-\Poss{}.\Third{}//
\glft `His/her/their elder sister.'//
\endgl
\xe

In Mongolian, the possessor can either bear genitive or be unmarked (mostly in idiomatic expressions). However, neither in \parref{mongola} nor in \parref{mongolb} can the possessive suffixes appear. This contrasts Mongolian with Even. Possessive markers in languages such as Mongolian are possessors themselves, and they are just cliticized to the head due to some phonetic rules. I suggest that the division I have made is crucial for understanding the nature of reflexive possessivity. The main interest is in languages of the first type (like Even), but some generalization (mostly typological and descriptive) can be made on the material from both groups.

\begin{table}[h!]
	\centering
	\small
	\begin{tabular}{@{}cccccc@{}}
		\toprule
		& juxtaposition & \parbox{2.5cm}{C-marker without possessor} & \parbox{2.75cm}{C-marker with unmarked possessor} & \parbox{1.75cm}{C-marker \& D-marker} & \parbox{3cm}{D-marker with unmarked possessum} \\ \midrule
		Mongolian & + & + & - & - & + \\ \bottomrule
	\end{tabular}
	\caption{Possessive marking strategies in Mongolian}
\end{table}

\subsection{Abaza}

Possession in Abaza is indicated by prefixes resembling those used for marking indirect objects in verbs and they occupy the same position as the definite marker. There is no distinction between alienable and inalienable possession, and both body-part and kinship nouns can appear without possessive prefixes when appropriate.

The relative prefix is utilized when the possessor argument is constrained by a question or relativization operator at the clause level, as seen in example (\ref{relative_prefix}).

\ex
\label{relative_prefix}
\begingl
\gla {áẑ-aĉ̣-kʷa} z-qa z-zə́-nqʷə-m-ga-χ-kʷa-wa//
\glb DEF+old-NPRO-PL REL.IO-head REL.IO-POT-LOC-NEG-carry-RE-PL-IPF //
\glft 'the old ones, who can no longer take care of themselves' //
\endgl
\xe

Possessive prefixes serve as the standard means of attaching referential modifiers to nouns (including nominalizations of verbs formally marked by suffixes). There is no restriction on inanimate possessors. The third person possessive prefixes are referential and do not necessitate an overt nominal possessor (\ref{referential_abaza}).

\pex
\a\label{referential_abaza}
\begingl
\gla awə́j á-dg’əl j-ájš’-ĉa-kʷa jə́-r-ĉ̣a-d //
\glb DIST DEF-land 3SG.M.IO-brother-PLH-PL 3SG.M.ERG-CAUS-sit(AOR)-DCL //
\glft 'He settled his brothers on this land.' //
\endgl
\xe

\begin{table}[h!]
	\centering
	\small
	\begin{tabular}{@{}cccccc@{}}
		\toprule
		& juxtaposition & \parbox{2.5cm}{C-marker without possessor} & \parbox{2.75cm}{C-marker with unmarked possessor} & \parbox{1.75cm}{C-marker \& D-marker} & \parbox{3cm}{D-marker with unmarked possessum} \\ \midrule
		Abaza & - & + & + & NA & NA \\ \bottomrule
	\end{tabular}
	\caption{Possessive marking strategies in Abaza}
\end{table}


\subsection{Russian}

Possession in Russian is indicated by possessive pronouns or genitive case marking. Possessive pronouns agree with the gender, number, and case of the possessed noun (\ref{rus_possessive_pronouns}), while the genitive case is invariant.

\ex
\label{rus_possessive_pronouns}
\begingl
\gla Eto {moya kniga} //
\glb This my.\F.\Nom{}.\Sg{} book(\F).\Nom.\Sg{} //
\glft 'This is my book.' //
\endgl
\xe

Genitive case marking is commonly used to indicate possession or ownership by showing the relationship between the possessor and the possessed noun, as in \ref{genitive_marking}.

\ex
\label{genitive_marking}
\begingl
\gla  kniga {Mash-i} //
\glb book(\F).\Nom.\Sg{} Masha-\Gen.\Sg{}//
\glft `Masha's book.' //
\endgl
\xe

Both strategies in Russian are dependent marking, and juxtaposition is impossible.

\begin{table}[h!]
	\centering
	\small
	\begin{tabular}{@{}cccccc@{}}
		\toprule
		& juxtaposition & \parbox{2.5cm}{C-marker without possessor} & \parbox{2.75cm}{C-marker with unmarked possessor} & \parbox{1.75cm}{C-marker \& D-marker} & \parbox{3cm}{D-marker with unmarked possessum} \\ \midrule
		Russian & - & NA & NA & NA & + \\ \bottomrule
	\end{tabular}
	\caption{Possessive marking strategies in Russian}
\end{table}

\subsection{Poqomam}

Possession is an inflectional category for nouns in Poqomam. Possessed nouns are inflected using Set A prefixes, which cross-reference the person and number of the possessor. Some nouns also exhibit a distinction between alienable and inalienable possession.

Possessor can be omitted, but when it is mentioned, it follows the possessed noun directly, as illustrated in example (\ref{possessor_mentioned}). 

\ex
\label{possessor_mentioned}
\begingl
\gla (ru-)malox kar//
\glb (A3-)egg fish//
\glft 'fish eggs' //
\endgl
\xe

In Poqomam, the Set A prefix for third person singular (A3) can sometimes be null (0) rather than \textit{ru-}, particularly when the possessed noun is an integral part of the possessor. This is common with relational nouns and may even be obligatory in some cases. It is hard to decide whether this behavior exemplifies juxtaposition or complex allomorphy. I treat this as former, however I admit that this assumption may be controversial.

\begin{table}[h!]
	\centering
	\small
	\begin{tabular}{@{}cccccc@{}}
		\toprule
		& juxtaposition & \parbox{2.5cm}{C-marker without possessor} & \parbox{2.75cm}{C-marker with unmarked possessor} & \parbox{1.75cm}{C-marker \& D-marker} & \parbox{3cm}{D-marker with unmarked possessum} \\ \midrule
		Poqomam & + & + & + & NA & NA \\ \bottomrule
	\end{tabular}
	\caption{Possessive marking strategies in Poqomam}
\end{table}


\subsection{Central Alaskan Yupik}

Possessed nouns agree in person and number with the possessors. While a third person (possessor) suffix entails an anaphoric reference, a reflexive-third person refers back to the third person subject of the main clause. The reflexive third person is required if it is the same person as the third-person subject.

A third-person possessor marking is obligatory in the head NP of attributive (genitive) phrases. It agrees in number with the dependent NP in the relative case, as shown in the following example, which adds a dependent NP to the structure:

\ex
\begingl
\gla May’a-m pani-a assik-aa.//
\glb p.n.-REL.sg. Da-ABS.3sg.sg. like-IND.3sg.3sg.//
\glft ‘He likes Mayaq’s daughter.’ //
\endgl
\xe

An independent personal pronoun can optionally be used to place emphasis on the possessor, as indicated in the following examples.

\pex
\a
\begingl
\gla aata-ka//
\glb ABS.1sg.sg.//
\glft ‘my father’ //
\endgl
\a
\begingl
\gla wiinga aata-ka//
\glb 1sg. ABS.1sg.sg.//
\glft ‘my father’ (emphasized) //
\endgl
\xe 

It must be noted that \textit{wiinga} `\Fsg' is not an unmarked form but rather a relative case form, which appears to coincide with an absolutive form.

\begin{table}[h!]
	\centering
	\small
	\begin{tabular}{@{}cccccc@{}}
		\toprule
		& juxtaposition & \parbox{2.5cm}{C-marker without possessor} & \parbox{2.75cm}{C-marker with unmarked possessor} & \parbox{1.75cm}{C-marker \& D-marker} & \parbox{3cm}{D-marker with unmarked possessum} \\ \midrule
		CAY & + & - & + & - & + \\ \bottomrule
	\end{tabular}
	\caption{Possessive marking strategies in Central Alaskan Yupik}
\end{table}

\subsection{Haida}

In Haida, there are essentially two formal types of alienable possession. One type includes the indefinite pronoun \textit{gyaa}, and the other does not. The type without \textit{gyaa} simply consists of the possessor NP with the suffix \textit{-ra} (South dialect) or \textit{-aa} (Masset dialect), for example (\ref{baskets}).

\ex\label{baskets}
\begingl
\gla Mary-ra qigw-aay tVl dahxu tVlraa-gan.//
\glb Mary-POSS basket-DF INDF buy all-PA//
\glft ‘People bought all Mary’s baskets.’ //
\endgl
\xe

There is another suffix \textit{-ra} (South dialect) or \textit{-aa} (Masset dialect), distinct from the previously mentioned one, that occurs on a lexically restricted class of possessed nouns. Typically, the possessor NP can be of any complexity and precedes the suffixed possessum without additional marking (\ref{guns_men}).

\ex\label{guns_men}
\begingl
\gla 7iihUnts ’id-aay 7waa dluu xan jigw-aara//
\glb be.men-NOM all gun-POSS//
\glft ‘the guns of all the men’ //
\endgl
\xe

The productive form of alienable possession consists of a sequence of NP \textit{gyaara} (South dialect) or NP \textit{gyaa} (Masset dialect) before the possessum (\ref{stolen_money}). 

\ex\label{stolen_money}
\begingl
\gla ra.agee 7waadluwaan gyaa daal-ee Joe q ’uhldaa-yaa-n.//
\glb the.children all POSS money-DF Joe steal-EVID-PA//
\glft ‘Joe stole every child’s money.’ //
\endgl
\xe

There are some argument to consider \textit{-ra}, \textit{-aa} found with \textit{gyaa} in the non-reflexive form to be the second \textit{-ra} that occurs on possessums. Consequently, the pronoun \textit{gyaa} is equivalent to the possessum: ‘Joe stole every child’s stuff, the money’.


\begin{table}[h!]
	\centering
	\small
	\begin{tabular}{@{}cccccc@{}}
		\toprule
		& juxtaposition & \parbox{2.5cm}{C-marker without possessor} & \parbox{2.75cm}{C-marker with unmarked possessor} & \parbox{1.75cm}{C-marker \& D-marker} & \parbox{3cm}{D-marker with unmarked possessum} \\ \midrule
		Haida & - & + & + & - & + \\ \bottomrule
	\end{tabular}
	\caption{Possessive marking strategies in Haida}
\end{table}

\subsection{To'abaita}

Possession in To'abaita is expressed through suffixing possessive noun phrases. The basic structure of the suffixing possessive noun phrase is given as follows: [(CLF) POSSESSUM.NOUN-PERS]$_{NP}$ ([lexical.possessor]$_{NP}$).

\ex
\label{suffixing_example}
\begingl
\gla maqa daar-a wane //
\glb CLF forehead-3.PERS man //
\glft ‘a/the man’s forehead’ //
\endgl
\xe

The only noun-phrase element that can occur with the head noun in the possessum noun phrase is a classifier. The possessum noun carries a personal suffix that indexes the possessor (\ref{suffixing_example}). Possessor noun phrases are optional, and they can only be lexical, not pronominal.

Another strategy is described as the bare possessive noun phrase. The basic structure of bare possessive noun phrases is given as follows: [... POSSESSUM.NOUN ...]$_{NP}$ [possessor]$_{NP}$ (\ref{bare_tobaq}).

\ex\label{bare_tobaq}
\begingl
\gla maka nau //
\glb father 1SG //
\glft ‘my father’ //
\endgl
\xe

In bare possessive noun phrases, there is no indexing of the possessor on the possessum noun. For that reason, a possessor noun phrase must be present. In the absence of a possessor phrase, a noun phrase cannot be interpreted as a possessive one. The possessor noun phrase may be lexical or pronominal.


\begin{table}[h!]
	\centering
	\small
	\begin{tabular}{@{}cccccc@{}}
		\toprule
		& juxtaposition & \parbox{2.5cm}{C-marker without possessor} & \parbox{2.75cm}{C-marker with unmarked possessor} & \parbox{1.75cm}{C-marker \& D-marker} & \parbox{3cm}{D-marker with unmarked possessum} \\ \midrule
		To'abaita & + & + & + & NA & NA \\ \bottomrule
	\end{tabular}
	\caption{Possessive marking strategies in To'abaita}
\end{table}

\subsection{West Coast Bajau}

The head noun can be followed by an optional possessive phrase. This phrase may include a possessive pronoun from set I, which attaches to the head noun if it is in an enclitic form, or it can be an independent NP (\ref{wcb_basics}).

\pex\label{wcb_basics}
\a
\begingl
\gla ruma’=ni //
\glb house=3s.I //
\glft ‘his home’ //
\endgl

\a
\begingl
\gla moto Deli //
\glb eye PN //
\glft ‘Deli’s eyes’ //
\endgl
\xe

\begin{table}[h!]
	\centering
	\small
	\begin{tabular}{@{}cccccc@{}}
		\toprule
		& juxtaposition & \parbox{2.5cm}{C-marker without possessor} & \parbox{2.75cm}{C-marker with unmarked possessor} & \parbox{1.75cm}{C-marker \& D-marker} & \parbox{3cm}{D-marker with unmarked possessum} \\ \midrule
		WCB & + & + & - & NA & NA \\ \bottomrule
	\end{tabular}
	\caption{Possessive marking strategies in West Coast Bajau}
\end{table}

\subsection{Iloko}

Iloko has two ways of expressing possession:
\begin{enumerate}
	\item Possessed noun followed by an NP designating the possessor.
	\item Possessed noun followed by an ergative pronominal enclitic.
\end{enumerate}

Normally, a possessor follows the possessed referent. Possessive constructions are done simply by juxtaposing possessum and possessor. A possessor may be encoded with a full noun phrase (the appropriate article introducing the relevant nominal possessor) or an ergative enclitic pronoun.

\pex
\a
\begingl
\gla ti baiay ni Maria  //
\glb ART house PA Maria  //
\glft 'Maria's house'  //
\endgl

\a
\begingl
\gla ti balay=na //
\glb ART house=3sE //
\glft 'her house' //
\endgl
\xe

\begin{table}[h!]
	\centering
	\small
	\begin{tabular}{@{}cccccc@{}}
		\toprule
		& juxtaposition & \parbox{2.5cm}{C-marker without possessor} & \parbox{2.75cm}{C-marker with unmarked possessor} & \parbox{1.75cm}{C-marker \& D-marker} & \parbox{3cm}{D-marker with unmarked possessum} \\ \midrule
		Iloko & + & + & - & NA & NA \\ \bottomrule
	\end{tabular}
	\caption{Possessive marking strategies in Iloko}
\end{table}

\subsection{Lundayeh}

In Lundayeh, possession can be indicated using two sets of pronouns: cliticized genitive and phonologically free possessive. The genitive pronouns come after the head, while the possessive pronouns precede the head (see Example (\ref{pronouns_lund})).

\pex\label{pronouns_lund}
\a
\begingl
\gla nan ticu’=kuh //
\glb on hand=1SG.GEN //
\glft `on my hand' //
\endgl

\a
\begingl
\gla diko anak//
\glb \Ssg.\Poss{} child//
\glft `your child'//
\endgl
\xe

In the examples, possessive pronouns are primarily used without a head in the predicative function. However, they can also express adnominal possession.

When the possessor is a noun phrase, juxtaposition is used (see example (\ref{nouns_lund})).

\ex\label{nouns_lund}
\begingl
\gla Momol-momol mo’ tang kuyad dih kuman bua’ timun ret nan on lati’ kidih. //
\glb REDUP-full.mouth PTCL mouth monkey that.REM eat.AV fruit cucumber from hand 1SG.REM farm 1SG.GEN //
\glft ‘The monkey’s mouth was very full, having eaten cucumbers from my farm.’ //
\endgl
\xe

\begin{table}[h!]
	\centering
	\small
	\begin{tabular}{@{}cccccc@{}}
		\toprule
		& juxtaposition & \parbox{2.5cm}{C-marker without possessor} & \parbox{2.75cm}{C-marker with unmarked possessor} & \parbox{1.75cm}{C-marker \& D-marker} & \parbox{3cm}{D-marker with unmarked possessum} \\ \midrule
		Lundayeh & + & + & - & - & + \\ \bottomrule
	\end{tabular}
	\caption{Possessive marking strategies in Lundayeh}
\end{table}


\subsection{Kobon}

In a general noun phrase where a noun functions as the attributive element, the relationship between the attribute and head may involve possession \ref{ex:kobon_juxt}. This is typically the only possessive construction, apart from the possibility of postposing a possessive pronoun to the nominal head in the general noun phrase.

\pex 
\a\label{ex:kobon_juxt}
\begingl
\glpreamble Kobon//
\gla Dumnab ram//
\glb Dumnab house//
\glft `Dumnab's house'//
\endgl

\a\label{ex:kobon_head}
\begingl
\gla Namam ne au-ab. //
\glb 2possbrother 2s come-pres3s //
\glft 'Your (singular) brother is coming.' //
\endgl
\xe

There is a distinction in the expression of possession concerning persons compared to animals and things. Most kinship terms carry prefixes and/or suffixes and/or undergo internal morphological changes to indicate the person (but not the number) of the possessor. In each case, the form is used for singular, dual, and plural number. Usually, the kinship term will be followed by the corresponding possessive pronoun too. The prefixes are 0- for the first person, na- for the second person, and no-/n±-/ni- for the third person \ref{ex:kobon_head}.

\begin{table}[h!]
	\centering
	\small
	\begin{tabular}{@{}cccccc@{}}
		\toprule
		& juxtaposition & \parbox{2.5cm}{C-marker without possessor} & \parbox{2.75cm}{C-marker with unmarked possessor} & \parbox{1.75cm}{C-marker \& D-marker} & \parbox{3cm}{D-marker with unmarked possessum} \\ \midrule
		Kobon & + & + & + & NA & NA \\ \bottomrule
	\end{tabular}
	\caption{Possessive marking strategies in Kobon}
\end{table}


\subsection{Hup}

Alienable possession in Hup is marked by the postpositional particle \textit{nɨ̐h} (which receives stress and a rising tone). This particle is associated with the possessor, phonologically so in the case of pronouns, and usually precedes the possessum, as shown in examples (\ref{hup}).

\pex\label{hup}
\a
\begingl
\gla tǽ=d’əh nɨ̐h, y-d’ə̌h nɨ̐h děh //
\glb ant.sp=PL POSS that-PL POSS water //
\glft `The water (saliva) of those ones, those tǽ ants.' //
\endgl

\a
\begingl
\gla tɨnɨ̐h mɔ̌y g’ǒd-ót, hib’ah-tæ̃h=ʔɨh nɨ̐h mɔ̌y g’ǒd-ót… //
\glb 3sg.POSS house inside-OBL created-son=MSC POSS house inside-OBL //
\glft `Inside his house, the created one’s house…' //
\endgl
\xe

Possessive forms in Hup can occur independently of a possessum, although this is relatively uncommon.

Inherently possessed nouns such as kinship terms, human nouns, and plant parts show another possessive strategy. This strategy involves so-called bound nouns. The possessor is attached to the possessum forming a compound (\ref{hup_compound}). The form of a noun or a pronoun is similar to the one used in the subject position.

\pex\label{hup_compound}
\a
\begingl
\gla núp tɨ́h=yãwám, pæ̃́y=wəd-ə́h //
\glb this 3sg=younger.brother thunder=RESP-DECL //
\glft ‘This was his younger brother, Full-of-Thunder.’ //
\endgl

\a
\begingl
\gla n’ip cidídu=tóg ham-ʔay-ní-h //
\glb that Cirino=daughter go-VENT-INFR2-DECL //
\glft ‘And that daughter of Cirino’s went and returned.’ //
\endgl
\xe

It is hard to decide whether compounds present a head-marking pattern or it is an instance of juxtaposition. According to the description, it seems more fair to consider them a head marking strategy.


\begin{table}[h!]
	\centering
	\small
	\begin{tabular}{@{}cccccc@{}}
		\toprule
		& juxtaposition & \parbox{2.5cm}{C-marker without possessor} & \parbox{2.75cm}{C-marker with unmarked possessor} & \parbox{1.75cm}{C-marker \& D-marker} & \parbox{3cm}{D-marker with unmarked possessum} \\ \midrule
		Hup & - & + & + & - & + \\ \bottomrule
	\end{tabular}
	\caption{Possessive marking strategies in Hup}
\end{table}


\subsection{Yucuna}

Within the NP, there is a distinct slot dedicated to the single argument of a head noun, or possessor. Nominal arguments in Yucuna are expressed through a rigid construction whereby the argument immediately precedes the head noun. Just like with postpositions and verbs, the argument of a head noun may appear in the form of an overt NP (\ref{ex:np}), free pronoun (\ref{ex:free_pronoun}), or bound person index (\ref{ex:bound_person_index}).

\pex \label{ex:np}
\begingl
\gla yáwi i'rí //
\glb tiger son //
\glft `tiger’s son' (ycn0053,10) //
\endgl

\a\label{ex:free_pronoun}
\begingl
\gla Ri=ikhá yukúná nu=i'ma-jé. //
\glb 3SG.NF=PRO story 1SG=tell-FUT //
\glft `I will tell his story.' //
\endgl

\label{ex:bound_person_index}
\begingl
\gla ri=i'rí //
\glb 3SG.NF=son //
\glft `his son'  //
\endgl
\xe

Beyond this general rule of possession encoding, there are important morphosyntactic differences that depend on the lexical class of the possessed noun. Indeed, nominal roots can be divided into three distinct possession classes: obligatorily possessed nouns, optionally possessed nouns, and non-directly possessible nouns. The two former classes allow the expression of a nominal argument, whilst the latter disallows it and requires a completely different possessive construction. 

Non-directly possessible nouns in this class disallow the Possessor Possessed juxtaposition construction described above. However, although semantically `unpossessible', these nouns may participate in a different possessive construction with the \textit{le'jé} POSS placed before the possessed noun (\ref{poss_construction}).

\ex \label{poss_construction}
\begingl
\gla ri=le'jé pají //
\glb 3SG.NF=POSS house //
\glft ‘his maloca (traditional house)’ //
\endgl
\xe

\begin{table}[h!]
	\centering
	\small
	\begin{tabular}{@{}cccccc@{}}
		\toprule
		& juxtaposition & \parbox{2.5cm}{C-marker without possessor} & \parbox{2.75cm}{C-marker with unmarked possessor} & \parbox{1.75cm}{C-marker \& D-marker} & \parbox{3cm}{D-marker with unmarked possessum} \\ \midrule
		Yucuna & + & + & - & - & + \\ \bottomrule
	\end{tabular}
	\caption{Possessive marking strategies in Yucuna}
\end{table}

\subsection{Kwaza}

The possessor -- the modifier in the Kwaza possessive construction -- is a personal pronoun or a noun. These constructions require a derivational possessive morpheme \textit{-dy-}, which must be applied to the possessor and which must be followed by a classifier, usually the nominaliser \textit{-hy}, which functions as a semantically neutral classifier, as exemplified in (\ref{buttock}).

\ex\label{buttock}
\begingl
\gla 'si-dy-hy ecui'ri //
\glb I-POS-NOM buttock //
\glft `my buttock' //
\endgl
\xe

Sometimes a choice is possible between a Neutral or a more specific classifier. When the specific classifier is etymologically related to the referent it classifies, the referent may be omitted, as in (\ref{omitted}), as if the classifier were a cross-reference morpheme.

\ex \label{omitted}
\begingl
\gla (a'xy) 'si-dy-xy //
\glb (house) I-POS-CL:house //
\glft `my house' //
\endgl
\xe

This is the only language in the sample that allows for omitting the possessum. From one hand, it could be analyzed in a way that the classifier morpheme is the possessum itself; however, it seems odd for me.
 
\begin{table}[h!]
	\centering
	\small
	\begin{tabular}{@{}cccccc@{}}
		\toprule
		& juxtaposition & \parbox{2.5cm}{C-marker without possessor} & \parbox{2.75cm}{C-marker with unmarked possessor} & \parbox{1.75cm}{C-marker \& D-marker} & \parbox{3cm}{D-marker with unmarked possessum} \\ \midrule
		Kwaza & - & NA & NA & NA & + \\ \bottomrule
	\end{tabular}
	\caption{Possessive marking strategies in Kwaza}
\end{table}

\printglosses[style=mcolblock]

\printbibliography
\end{document} 
