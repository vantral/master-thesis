\documentclass[a4paper, 12pt]{article}
\usepackage[left=2.5cm,right=2.5cm,
            top=2.5cm,bottom=2.5cm,
            bindingoffset=0cm]{geometry}

% useful package for XeLaTex
\usepackage{xltxtra}

% some specifications for math text
\defaultfontfeatures{Ligatures=TeX,Mapping=tex-text}
\usepackage[LGRgreek,noendash]{mathastext}
\usepackage{amsmath}

% setting default fonts
\usepackage{fontspec}
\setmainfont{Brill-Roman.ttf}[
 ItalicFont = Brill-Italic.ttf,
 BoldFont = Brill-Bold.ttf,
 BoldItalicFont = Brill-BoldItalic.ttf]
\setsansfont{FreeSans}
\setmonofont{FreeMono}

% ipa symbols if needed
\usepackage{tipa}


% babel
\usepackage[english]{babel}


% tikz for images and forest for linguistic trees
\usepackage{tikz}
\usetikzlibrary{calc}
\usetikzlibrary{positioning}
\usepackage[linguistics]{forest}

% library for nice arrows
\usetikzlibrary{paths.ortho}


% styles enumeration
\usepackage[shortlabels]{enumitem}


% gloss formatting
\usepackage{expex}
\lingset{everygla=\normalfont, belowglpreambleskip=-6pt, aboveglftskip=-6pt, belowexskip=0pt, aboveexskip=0pt, glhangstyle=none, extraglskip=!.25ex}


% glossing
\usepackage{leipzig}
% automatic small caps
\renewcommand{\leipzigfont}[1]{\textsc{#1}}
% input file with glosses for leipzig package
\newleipzig{Coref}{coref}{coreferent}
\newleipzig{Npst}{npst}{non-past}
\newleipzig{Ep}{ep}{epenthesis}
\newleipzig{Inch}{inch}{inchoative}
\newleipzig{Ss}{ss}{same subject}
\newleipzig{Todp}{tod.p}{today's past tense}
\newleipzig{S}{s}{singular}
\renewleipzig{Ptcp}{ptc}{participle}
\newleipzig{Nfut}{nfut}{non-future}
\newleipzig{Dir}{dir}{directive}
\renewleipzig{Incl}{inc}{inclusive}
\renewleipzig{Excl}{exc}{exclusive}
\newleipzig{Prol}{prol}{prolative}
\newleipzig{Dloc}{dloc}{directive-locative}
\newleipzig{Add}{add}{additive}
\newleipzig{Freq}{freq}{frequentative}
\newleipzig{Dim}{dim}{diminutive}
\newleipzig{ConvAnt}{convant}{~~~~~~anterior converb}
\newleipzig{Sup}{sup}{supine}
\newleipzig{Pr}{pr}{~}
\newleipzig{And}{and}{~}
\newleipzig{Impf}{impf}{imperfective}
\newleipzig{Nsg}{nsg}{non-singular}
\newcommand{\Fnsg}{\First{}\Nsg{}}
\newleipzig{A}{a}{absolutive}
\newleipzig{E}{e}{ergative}
\newleipzig{Nonactive}{nonactive}{non-active}
\newleipzig{Clitic}{clitic}{clitic}
\newleipzig{Nw}{nw}{nonwitnessed}
\newleipzig{Wp}{wp}{witnessed past tense}
\newleipzig{V}{v}{\textsc{v} class in Ingush}
\newleipzig{J}{j}{\textsc{j} class in Ingush}
\newleipzig{I}{i}{first gender in Archi}
\newleipzig{II}{ii}{second gender in Archi}
\newleipzig{III}{iii}{third gender in Archi}
\newleipzig{IV}{iv}{fourth gender in Archi}
\newleipzig{Evid}{evid}{evidentiality}
\newleipzig{Int}{int}{intensifier particle}
\newleipzig{Atr}{atr}{attributivizer}
\newleipzig{Ill}{ill}{illative}
\newleipzig{Trans}{trans}{transitional}
\newleipzig{Compl}{compl}{completion}
\newleipzig{In}{in}{inessive}
\newleipzig{Fst}{fst}{false start}
% \newleipzig{}{}{}
% \newleipzig{}{}{}
% \newleipzig{}{}{}
% \newleipzig{}{}{}
% list of abbreviations
\usepackage{glossaries}
\makeglossaries


% adding appendix
\usepackage[toc,page]{appendix}


% styling hyperlinks
\usepackage{hyperref}
\hypersetup{
 colorlinks,
 citecolor=[RGB]{34, 37, 117},
 filecolor=black,
 linkcolor=[RGB]{34, 37, 117},
 urlcolor=[RGB]{34, 37, 117},
 pdfborder={0 0 0}
}

% citation format
\usepackage[parentracker=true,
backend=biber,natbib,
hyperref=true,
bibencoding=utf8,
maxcitenames=2,
style=biblatex-gl,
citestyle=gl-authoryear-comp
]{biblatex}

% cite year command
\DeclareCiteCommand{\citeyear}
    {}
    {\bibhyperref{\printfield{year}}\bibhyperref{\printfield{extradate}}}
    {\multicitedelim}
    {}

% cite year in parentheses
\DeclareCiteCommand{\citeyearpar}
    {}
    {\mkbibparens{\bibhyperref{\printfield{year}}\bibhyperref{\printfield{extradate}}}}
    {\multicitedelim}
    {}

% citation in parentheses
\newcommand{\parref}[1]{(\ref*{#1})}


% quotation
\usepackage{csquotes}


% matrices
\usepackage{array}
% vertical spacing between rows in arrays
\renewcommand{\arraystretch}{1.25}


% adding multicolumn mode
\usepackage{multicol}
\setlength{\columnsep}{-0.5cm}


% nice and long tables
\usepackage{booktabs}
\usepackage{longtable}

% captions for subfigures and subtables
\usepackage{subcaption}

% default bib resource
\addbibresource{bib.bib}


% default spacing set to 1.5, no parindent,
% spacing after paragraphs is 6pt
\usepackage{setspace}
\onehalfspacing
\setlength{\parindent}{0pt}
\setlength{\parskip}{6pt}

% list of abbreviation formating
\renewcommand{\glossarypreamble}{\hfill\begin{minipage}{\textwidth}}
\renewcommand{\glossarypostamble}{\end{minipage}\hfill}

% special apostrophe
\renewcommand{\'}{\textquotesingle}

\usepackage{multirow}

\singlespacing

\author{Anton Buzanov}
\title{Possessive Marking: Dependent and Non-dependent marking}
\date{\today}

\begin{document}
\maketitle

\section{Introduction}

This thesis is dedicated to possessive constructions. Within this study, possessive constructions are defined as comprising two noun phrases: the possessor and the possessum. Their primary function is to convey possessive relationships, which are broadly interpreted. Among these relationships, ownership stands out as the most common.

To illustrate these concepts, several examples of possessive constructions are provided in \parref{primary_examples}. These examples are to clarify the core focus of the study. Furthermore, they highlight the ways in which possessive relationships can be marked within noun phrases. For instance, while the Russian example indicates possession solely through the genitive case on the dependent, the Even example employs a possessive suffix on the head to convey the possessive relationship.

\pex
\label{primary_examples}
\a
\begingl
\glpreamble Russian (Indo-European)//
\gla dom-\o{} otc-a//
\glb house-\Nom.\Sg{} father-\Gen.\Sg{}//
\glft `father\'s house'//
\endgl 

\a
\begingl
\glpreamble Even (Tungusic)//
\gla etiken  ǯu-n//
\glb old.man house-\Poss.\Tsg{}//
\glft `old man's house'//
\endgl 
\xe

Various types of possessive constructions can be identified. For instance, in Russian, there exists another construction primarily used to express possessive relationships, as detailed in \parref{possessive_adjective}. Forms such as \textit{mam-in-a} are commonly referred to as "possessive adjectives" (\textit{pritjažatel'nye prilagatel'nye} in Russian). This terminology suggests that possessors in these constructions behave more like adjectives than proper nouns. They inflect in oblique cases as adjectives and cannot be modified by other adjectives.

Constructions where either the possessor or the possessum deviates from typical noun behavior were excluded from the analysis. This exclusion criterion is relevant primarily in languages with a clear distinction between nouns and adjectives. The focus of this study remains exclusively on the interaction between noun phrases.

\ex
\label{possessive_adjective}
\begingl
\glpreamble Russian (Indo-European)//
\gla mam-in-a kniga//
\glb mother-\Poss.\Adj-\Nom.\Sg.\F{} book//
\glft `mother\'s book'//
\endgl
\xe

The distinction between alienable and inalienable possession \citep{nichols1988alienable,alexiadou2003some} is not the central focus of this thesis. Both types of constructions are treated with equal importance.

The primary objective of this study is to develop a typology for possessive marking across various language types. Specifically, the focus lies on morphosyntactic patterns such as the locus of marking, obligatoriness, and agreement.

The thesis is structured as follows:

Chapter \ref{litreview} provides an overview of previous approaches to describing possessive constructions. Chapter \ref{problems} identifies the shortcomings in these approaches and presents sporadic data that highlights the missed important generalizations. In this chapter, I introduce my own classification system for languages based on how they express possessive relations.

Chapter \ref{data} presents data from various languages to support the plausibility of my classification system over previous ones. Within this chapter, I highlight the differences between dependent and non-dependent marking within noun phrases.

Chapter \ref{asymmetries} articulates the asymmetries between dependent and non-dependent possessive marking that I identified during my study. 

Finally, Chapter \ref{conclusion} draws conclusions based on the findings presented in the preceding chapters.

\section{Previous Studies}
\label{litreview}

Possessive constructions have been described extensively for individual languages (references), as well as in cross-linguistic studies (references). However, it is noteworthy that most cross-linguistic investigations have focused primarily on alienability, existential, or locative possessives, or solely on possessive constructions marked on the dependent (references).

\subsection{Locus of Marking in Possessive Noun Phrases}

The literature on the locus of marking, particularly as discussed by \cite{nichols1986head} and later scholars, provides perhaps the most neutral typology of possessive noun phrases. \cite[57]{nichols1986head} argues that the locus of marking does not signify any distinction in the syntactic relation, which remains consistent across different marking patterns.

Based on the specific locus of marking within possessive noun phrases, languages can be distinctly categorized into several groups \citep{nichols_locus_2013, van2016locus, van2016grammaticalization, lander2020head}.

The types of possessive marking within a possessive noun phrase, as distinguished by \cite{nichols_locus_2013}, along with the respective numbers of languages for each type in brackets, are as follows.

\begin{enumerate}
    \item Head marking (78)
    \item Dependent marking (98)
    \item Double marking (22)
    \item No marking (32)
    \item Other (6)
\end{enumerate}

However, this classification seems problematic for several reasons. A later revision of this categorization was presented in \cite{lander2020head}. In this revised classification, \cite{lander2020head} introduced D- and C-marking, corresponding to dependent marking and anywhere-but-dependent marking. This distinction enables the classification of various cases initially placed in the \textit{Other} category.

To better understand the labels employed by \cite{nichols_locus_2013}, and where necessary, incorporating insights from \cite{lander2020head}, I will describe their meanings.

\subsubsection{Head marking}

\begin{quote}
    In these examples the possessed noun (the head) agrees in person and number with the possessor noun, the most common pattern for head-marked noun phrases. Agreement in gender is also fairly common. A few languages have a non-agreeing marker on the head noun. In Fijian, a possessive affix -i marks possessed nouns; that it does not vary for person or number of the possessor. \citep{nichols_locus_2013}
\end{quote}

\pex\label{headmarking}
\a
\begingl
\glpreamble Acoma (Keresan; \citealt[177]{Miller-1965} from \citealt{nichols_locus_2013}) //
\gla s'adyúm'ə gâam'a//
\glb \Fsg.brother \Tsg.house//
\glft `my brother's house' (lit. `my-brother his-house')//
\endgl
\a
\begingl
\glpreamble Fijian (Austronesian; \citealt[36]{Dixon-1988} from \citealt{nichols_locus_2013}) //
\gla a liga-i ‘eirau//
\glb \Art{} hand-\Poss{} \Fdu.\Excl{}//
\glft `our hand(s)'//
\endgl

\xe 

In the examples (\ref{headmarking}), the possessed entity receives a distinctive marker that serves to signify (i) its role as the possessed item and/or (ii) $\phi$-features, encompassing person, number, and gender, of the possessor. If this marker indicates both (i) and (ii), it is recognized as the equivalent of agreement between a verb and its argument, albeit occurring within the nominal domain. In the framework of \cite{lander2020head}, the representation of features related to the possessed entity is termed \textit{indexation} in contrast to \textit{registration}.

Notably, the possessor in this context remains morphologically unmarked.

While \cite{nichols_locus_2013} categorizes non-agreeing markers, they do not delve into the specific characteristics of these markers, aiming instead to group them together with agreeing markers.

\subsubsection{Dependent marking}

\begin{quote}
    In the following examples from Chechen (Nakh-Daghestanian), the possessor noun is in the genitive case. \citep{nichols_locus_2013}
\end{quote}


\pex\label{dependentmarking}
\glpreamble Chechen (\citealt{nichols_locus_2013}) //
\a
\begingl
\gla loem-an k'orni//
\glb lion-\Gen{} baby.animal//
\glft `lion cub’, `lion's cub’ (lit. `of-lion cub’)//
\endgl
\a
\begingl
\gla mashien-an maax//
\glb car-\Gen{} price//
\glft `the price of a car' (lit. `of-car price')// 
\endgl
\xe

In (\ref{dependentmarking}), only the possessor is marked with a special marker, while the possessum is marked with an externally governed case.

\subsubsection{Double marking}

\cite{nichols_locus_2013} does not provide specific commentary on double-marking constructions. Essentially, these constructions involve a combination of the two preceding types, where both the possessor and the possessum are marked with distinctive indicators that signify (i) their respective roles and/or (ii) the features of the other \parref{doublemarking}.

\ex
\label{doublemarking}
\begingl
\glpreamble Southern Sierra Miwok (Miwok-Costanoan; \citealt[133]{Broadbent-1964})//
\gla cuku-ŋ hu:ki-ʔ-hy://
\glb dog-\Gen{} tail-\Tsg//
\glft `dog's tail’ (lit. `of-dog its-tail’) //
\endgl
\xe

In example \parref{doublemarking}, both the genitive marker on the possessor and the possessive suffix on the possessum are present.

\subsubsection{No marking}

No special comments are made on no-marking constructions.

\ex
\begingl
\glpreamble Asmat (Asmat family; \citealt[136, 133]{Voorhoeve-1965b} from \citealt{nichols_locus_2013}//
\gla no cem//
\glb \Fsg{} house//
\glft `my house' (lit. `I house' or `me house')//
\endgl
\xe

Here, the possessor and the possessum are morphologically unmarked.

\subsubsection{Key Principles of \cite{nichols_locus_2013} Typology}

In their classification, \cite{nichols_locus_2013} select one construction per language, aiming for minimal restrictions within that construction. \cite{nichols_locus_2013} explicitly excludes constructions from their classification that do not permit overtly expressed possessors.

Furthermore, they assert that morphologically unmarked possessors exhibit distinct behavior compared to their marked counterparts. Their focus is solely on an open-class set of possessors, excluding consideration of personal pronouns.

\cite{lander2020head}, building on the work of others, exploit the concepts of \textit{registration} and \textit{indexation}. While \cite{nichols_locus_2013} does not explicitly address these concepts in their typology, for their purposes, it remains inconsequential.

\subsection{Registration and Indexation}

Another significant aspect that I will address is the distinction between indexation and registration \citep{nichols1992linguistic}. Indexing markers reflect certain $\phi$-features of another part of the construction, while registering markers simply indicate the presence of another part. Originally, this notion aimed to differentiate between two types of head-marking possessive constructions. However, \cite{lander2020head} extend this perspective to encompass dependent marking as well.

Similar concepts, albeit under different names, have been discussed by several other authors (Creissels 2017, Plank 1995).

In a related vein, \cite{duguine2008structural} explore this distinction in various languages. They demonstrate that possessors triggering agreement can be pro-dropped and extracted, whereas possessors that do not trigger agreement cannot.

Although these concepts are mentioned by various authors within different frameworks, as far as I am aware, there is no unified account or set of generalizations based on the nature of a marker.


\section{Problems with \cite{nichols_locus_2013} Typology}

Given my exclusive familiarity with the possessive construction typology presented by \cite{nichols_locus_2013}, I will compare my observations directly to this framework to enrich its scope. It's worth noting that my study does not concern itself with semantics or alienability, rendering typologies based on these factors irrelevant to my research.

I contend that the assertions made by \cite{nichols_locus_2013} are erroneous. Firstly, there is no necessity to exclude constructions lacking an overt possessor; in fact, their inclusion is essential for establishing a comprehensive typology of locus marking. Secondly, the postulation of a distinction between morphologically marked possessors, such as those bearing the genitive case, and morphologically unmarked genitives is unnecessary.

\subsection{Null Possessors}

I argue that what is called by \cite{nichols_locus_2013,lander2020head} a head-marking strategy is not a proper head-marking strategy but rather a double marking one.

The ideal example of head marking is that presented in (\ref{mongolian}).

\pex\label{mongolian}
\glpreamble Mongolian (Mongolic) from \cite{janhunen_mongolian_2012}//
\a\label{mosthead}
\begingl
\gla (*min-ii) duu-men'//
\glb \Fsg-\Gen{} younger.brother-\Poss.\First//
\glft `my younger brother'//
\endgl
\a
\begingl\label{mostdependent}
\gla min-ii eej(-*men')//
\glb \Fsg-\Gen{} mother-\Poss.\First//
\glft `my mother'//
\endgl
\xe

In (\ref{mongolian}), the simultaneous use of both genitive and possessive marking is precluded. This limitation can be elucidated by tracing the grammaticalization trajectory of possessive markers, originally postposed clitics derived from personal pronouns. In the current state, these markers are fully integrated phonologically, displaying no distinguishing features from regular affixes in Mongolian \citep[137]{janhunen_mongolian_2012}.

Example (\ref{mosthead}) exemplifies the perfect head-marking pattern, since there is no dependent to be marked. Conversely, example (\ref{mostdependent}) illustrates the ideal dependent-marking pattern.

According to the classification by \cite{nichols_locus_2013}, Mongolian is placed in the category of languages exhibiting dependent marking in possessive phrases. This is due to the fact that \cite{nichols_locus_2013} exclude constructions without overtly expressed possessors. However, it is evident to me that Mongolian must differ in this aspect from languages like Russian.

\subsection{Unmarked vs Marked Possessors}

Classic instances of head-marking possessors include those in (\ref{headmarking}) or the one referenced in (\ref{hung}). In such examples, the possessor remains morphologically unmarked.

\ex\label{hung}
\begingl
\glpreamble Hungarian adapted from \cite[263]{szabolcsi1981possessive}//
    \gla az én kar-ja-i-m//
    \glb the I arm-\Poss-\Pl-\Fsg//
    \glft `my arms'//
\endgl
\xe

Now, I want to draw the attention to the data of Even (Tungusic) cited in (\ref{even_basic_examples}).

\pex
\glpreamble Even (Tungusic), field data//
\a
\begingl
\gla bi əm-ni-wu//
\glb I.\Nom{} come-\Pst-\Fsg//
\glft `I came.'//
\endgl

\a
\begingl
\gla min ǯu-wu//
\glb I.\Obl{} house-\Poss.\Fsg//
\glft `my house'//
\endgl

\a
\begingl
\gla etiken əm-ni-n//
\glb old.man come-\Pst-\Tsg//
\glft `The old man came.'//
\endgl

\a
\begingl
\gla etiken ǯu-n//
\glb old.man house-\Poss.\Tsg//
\glft `old man's house'//
\endgl
\xe

These examples bear resemblance to Hungarian in that ordinary nouns do not exhibit special genitive marking. However, a distinction arises between Even and Hungarian concerning possessive forms with pronouns. Notably, Even demonstrates a distinction from Hungarian by featuring distinct possessive forms for pronouns.

While it may initially seem that \cite{nichols_locus_2013} exclusively focus on constructions with nominal possessors, Evenki, which is related to and behaves similarly to Even in this particular aspect, is categorized as a \textit{Double marking} language while Hungarian is categorized as a \textit{Head-marking} one. The primary rationale for this classification appears to be the presence of special possessive pronouns in Evenki. In the possessor position, both Evenki and Even employ an oblique stem, aligning with the direct stem for nouns and diverging from it for pronouns.

Indeed, when considering only nouns, Hungarian and Evenki appear similar, as both languages employ morphologically unmarked forms for possessors. However, despite this similarity, they are classified differently. An additional factor contributing to this discrepancy could be the presence of an obsolete genitive marker that occasionally appears on possessors in Evenki. This historical remnant might influence the classification of Evenki, leading to its categorization distinct from Hungarian despite their superficial similarity in possessive marking.

\subsection{Solutions}

I believe that aforementioned problems with classification of Mongolian, Hungarian and Evenki straightforwardly follows from wrong assumptions made by \cite{nichols_locus_2013}.

I contend that classifying languages or constructions as a whole is an inaccurate approach. A more effective method involves employing a bottom-up strategy, initially categorizing markers as either C- or D-marking \citep{lander2020head}, and subsequently classifying a language as a collection of pairs of these markers. For instance, in this framework, Russian would be characterized as a language featuring several D-markers, such as \Gen{} and possessive forms of (pro)nouns, while lacking any C-markers. This approach resembles the earlier concept of a dependent-marking language but offers a more flexible and nuanced classification.

In my second argument, I posit that treating languages with unmarked possessors differently from those with marked possessors a priori is an inaccurate approach. It is crucial to distinguish between zero marking and the absence of marking. The concept of the locus of marking presupposes the existence of a head and a dependent, with some form of relationship between the two. I propose that in languages featuring a grammaticalized case system, the possessor is indeed marked, but the marker itself is represented by zero.

This perspective might be seen as controversial, yet several arguments support this viewpoint. Consider languages like Even and Evenki, where a position is marked with the genitive, but only for personal pronouns. In such instances, it is reasonable to assume that for nouns, marking is also present, albeit in the form of zero. Moreover, adpositions in Even and Evenki also governs this unmarked form.

In the second argument, consider languages with declension classes where certain classes exhibit a coincidence between the nominative and genitive forms, as seen in Latin with \textit{avis} meaning 'bird.\Nom{} / bird.\Gen{}'. In such cases, determining whether these forms are marked or unmarked becomes ambiguous. On the one hand, the nominative is typically considered an unmarked case, even though it is morphologically marked. From a paradigmatic perspective, it might be more appropriate to label these forms as marked with the genitive rather than with the nominative.

In a language where nouns lack dedicated genitive forms and the form coinciding with the nominative is employed in possessor contexts, this shared nominative/genitive form is morphologically unmarked. This observation supports the idea of unmarked case. Essentially, it is a matter of cross-linguistic variation which form must be used as a default nominal dependent.

We must consider morphologically unmarked possessors in languages that have cases to be an instance of head-marking / double-marking strategy. In the same way, we consider unmakred nominative with agreeing verb to present double-marking strategy at the clause level.

\subsection{Summary}



\section{Hypotheses}

\subsection{Non-possessive meanings}

\subsubsection{Formulation}
\textbf{Possessive suffixes more often develop non-possessive meanings}, such as marking topic.

\subsubsection{Pipeline}

\section{Sampling for Variety}

To comprehend the nature and potential diversity of possessive constructions, particularly in the context of head and dependent marking interaction, constructing a variety sample is essential. The aim of a variety sample is to capture a broad spectrum of linguistic patterns with minimal effort, as outlined by \cite{miestamo2016sampling} following \citealt{rijkhoff1993method}.

Another approach to sampling is probability sampling, which allows for statistical generalizations based on language data. However, as discussed in the literature on language sampling, achieving a completely unrelated or independent sample is practically unattainable. In any large language sample, there are bound to be relationships among languages, whether they are genetic, areal, cultural, or otherwise interconnected.

To understand if there is a tendency to express something in a particular way, it is impractical to examine all languages of the world for several reasons. First, it is practically impossible given the large number of languages. Second, such a comprehensive survey would not necessarily yield accurate conclusions. \cite{dryer1989large} illustrated this point through a thought experiment. Imagine a hypothetical world consisting of 1000 languages, categorized as follows:

\begin{itemize}
	\item 900 languages originate from a single language family.
	\item The remaining 100 languages originate from ten different families.
\end{itemize}

Now, suppose in this world, the majority of languages (900 out of 1000 originating from a single family) exhibit SVO word order, while the remaining 100 languages demonstrate SOV word order. Despite observing a prevalence of SVO word order, we cannot conclusively state that there is an overall tendency towards SVO word order across languages. This observation is likely influenced by genealogical bias, where the high occurrence of SVO word order in the majority of languages can be attributed to their shared linguistic ancestry within the same family. Therefore, drawing generalizations about linguistic tendencies requires careful consideration of language relationships to avoid misleading interpretations based solely on observed patterns in a diverse linguistic landscape.


\citet{dryer1989large} and \citet{bickel2008refined}, among others, proposed methods for creating a probability sample in linguistic studies. \citet{dryer1989large} introduced the concept of ``genus'' to mitigate genealogical bias. A genus refers to a grouping of languages with a time depth not exceeding 3,500 to 4,000 years \citep{dryer1989large}. These genera are considered sufficiently independent to be included in the sample. Another criterion for independence is the distant geographic location of languages. \citet{dryer1989large} argues that so-called Macro-Areas (Africa, Eurasia, Australia + New Guinea, North America, South America) are independent from each other in terms of linguistic properties. The main idea is to assign values to all genera (allowing for multiple values since languages within a genus can exhibit different behaviors) and then create a table similar to Table \ref{dryer_table}.

% Please add the following required packages to your document preamble:
% \usepackage{booktabs}
\begin{table}[ht]
	\centering
	\begin{tabular}{@{}ccccccc@{}}
		\toprule
		\multicolumn{1}{l}{} & \textbf{Africa} & \textbf{Eurasia} & \textbf{Australia-NG} & \textbf{North America} & \textbf{South America} & \textbf{Total} \\ \midrule
		\textbf{SOV}         & 22              & 26               & 19                    & 26                     & 18                     & 111            \\
		\textbf{SVO}         & 21              & 19               & 6                     & 6                      & 5                      & 57             \\ \bottomrule
	\end{tabular}
	\caption{Word order in different genera across Macroareas}\label{dryer_table}
\end{table}

Next, \citet{dryer1989large} determines the prevalence of SOV or SVO word order within each Macroarea. Modeling this scenario as a binomial distribution, he suggests that there is a tendency towards SOV, reasoning that the probability of SVO ``winning'' five times (with a probability of 0.5 for each outcome) is less than 0.05 ($0.5^5 = 0.03125$), indicating a statistically significant preference for SOV word order.

Certainly, I can revise that. Here's an alternative wording:

While the methodology involving genera and macroareas may raise questions, the concepts have been employed in other studies dedicated to language sampling. \citet{miestamo2016sampling} proposed a method for creating a variety sample based on genera and macroareas, following the framework established by \citet{dryer1989large} regarding the relative independence of genera and macroareas. In variety sampling, complete independence is not mandatory. \citet{miestamo2016sampling} divided the world into six macroareas, with a distinct categorization for Australia and New Guinea. The Genus-Macroarea method involves selecting languages from a macroarea in proportion to the number of genera within that macroarea, aiming to achieve a representative and diverse sample of languages across different geographic and genetic classifications.



\subsection{}


\section{Other questions}

- How do possessive markers undergo grammaticalization? (Нужно почитать джоанну)

- Regarding the relative order of case and possessive markers: Dékány?

- Is there a relationship between the markedness of the nominative form and the relative order of case and possessive markers?

- Do marking strategies at the clause level, especially considering pro drop, correlate with strategies within NP? Verification of Nichols' observations.

- Do markers on the dependent reflect registration while markers on the head reflect indexing (or do they exhibit no preference)? What does this imply about the "universal" structure of NP?

- Are there any significant asymmetries between verbal agreement and nominal agreement? I'm aware of one epiphenomenal difference. It's difficult to find instances where Agreement failure elicits default agreement, as often occurs with sentential subjects in many languages. It's challenging to imagine a scenario where a non-nominalized clause could occupy the possessor's position.


\section{Possessive Systems}
\subsection{Bystraja Even}

Northern Tungusic < Tungusic

Word order: SOV, Pr Ps

Nominal affix order: stem-AL-Number-Case-Poss

Nominal possessive markers can be categorized into two subsets: personal and reflexive. Personal possessive markers indicate specific person values, including first (inclusive and exclusive in plural), second, and third persons. Reflexive possessive markers, as discussed by Buzanov (2022), are employed when there is co-reference between the possessor of a noun and the subject of the entire clause. While reflexive possessive markers are used with possessors of any person value, they do not differentiate between different person values. Both personal and reflexive markers distinguish between two number values: singular and plural. The paradigm for these markers is illustrated in Table \parref{even_poss}.

\begin{table}[ht]
\caption{Possessive markers in Even}\label{even_poss}
\begin{subtable}[t]{0.48\textwidth}
    \centering
    \caption{\centering Personal possessive suffixes}
    \label{perposs}
    \begin{tabular}[t]{ll}\addlinespace\toprule
        \Poss    & morpheme  \\\midrule
        \Fsg     & -wu   \\
        \Ssg     & -š(i)     \\
        \Tsg     & -n(i)       \\
        \Fpl.\Incl & -t(i)     \\
        \Fpl.\Excl & -wun \\
        \Spl     & -šan      \\
        \Tpl     & -tan      \\\bottomrule
    \end{tabular}
\end{subtable}
\hfill
\begin{subtable}[t]{0.48\textwidth}
    \centering
    \caption{\centering Reflexive possessive suffixes}
    \label{reflposs}
    \begin{tabular}{cl}\addlinespace\toprule
        \Poss.\Refl & morpheme\\\midrule
        \Sg & -i/-ji/-mi/-bi \\
        \Pl & -wur\\\bottomrule
    \end{tabular}
\end{subtable}
\end{table}

Another way of expressing possessive relations in Even is through the possessive form of a (pro)noun. Both personal pronouns and the reflexive pronoun in Even have distinct possessive forms, which I will refer to as the genitive form. These forms can be utilized within possessive constructions alongside the corresponding possessive suffixes detailed in Table \parref{even_poss}, or within postpositional phrases, which should be analyzed as possessive phrases.

The genitive stem is not only used for possessive forms but also to derive other case forms of pronouns, as illustrated in Table \parref{even_pronouns}. Unlike pronouns, nouns do not have special genitive forms. Nouns in the possessor position remain morphologically unmarked, much like their appearance in subject positions. Third-person pronouns exhibit noun-like behavior, evident in their morphological structure, which includes distinct non-suppletive exponents for number and possessive marking.

The first-person inclusive pronoun \textit{mut} also displays similar forms in both subject and possessor positions, which is unexpected.

\begin{table}[t]
    \centering
    \begin{tabular}[t]{llll}\addlinespace\toprule
        pronoun    & \Nom & \Gen & \Dat  \\\midrule
        \Fsg     & bi & min &  min-du  \\
        \Ssg     & i & in & in-du    \\
        \Tsg     & noŋa-n `(s)he-\Poss.\Tsg' & noŋa-n & noŋan-du-n `(s)he-\Dat-\Poss.\Tsg' \\
        \Fpl.\Incl & mut & mut & mut-tu \\
        \Fpl.\Excl & bu & mun & mun-du \\
        \Spl     & u & un & un-du     \\
        \Tpl     & noŋa-r-tan `(s)he-\Pl-\Poss.\Tpl' & noŋa-r-tan  & noŋa-r-du-tan `(s)he-\Pl-\Dat-\Poss.\Tpl'\\\bottomrule
    \end{tabular}
    \caption{\centering Personal pronouns in Even}
    \label{even_pronouns}
\end{table}

It is important to note that while possessive suffixes are obligatory, these genitive forms are optional (see \ref{even_optional}).

\ex
\label{even_optional}
\begingl
\gla min ǯu-wu//
\glb I.\Obl{} house-\Poss{}.\Fsg//
\glft `my house'//
\endgl
\xe 

In Even, compounds are absent. Instead, compound-like structures are conveyed through possessive phrases, exemplified in \parref{even_berry}.

\ex
\begingl
\gla munnukan təwtə-n//
\glb hare berry-\Poss.\Tsg//
\glft `cranberry' (lit. `hare\'s berry')//
\endgl
\xe

\subsubsection*{Summary on Even}

In Even, there are markers that appear on the head of possessive phrases, along with the genitive form of (pro)nouns, which is an instance of dependent marking.

While dependent marking is optional, head marking is obligatory and resemble agreement, occurring even with non-referential possessors.

\subsection{Kildin Saami}

Saami < Uralic

Word order: SVO, Pr Ps

Nominal affix order: stem-OmnPl-Case+Number-Poss

According to the possessive paradigms presented in \cite{kuruch_saamsko-russkij_1985} and \cite{kert_saamskij_1971}, there are three distinct possible person values as illustrated in Tables \parref{saami_poss1} and \parref{saami_poss2}. These scholars do not mention any variance between speakers. However, \cite{riesler_kildin_2022} notes that ``Kildin Saami has lost the regular possessive inflection of nouns. Remnants of the former possessive inflection are only found occasionally with kinship nouns".

\begin{table}[ht]
\centering
\begin{tabular}{lllllll}
\toprule
\multicolumn{1}{c}{\multirow{2}{*}{Case}} & \multicolumn{3}{c}{\Sg{} possessor, \Sg{} possessum} & \multicolumn{3}{c}{\Pl{} possessor, \Sg{} possessum}\\
\multicolumn{1}{c}{} & 1 person & 2 person & 3 person & 1 person & 2 person & 3 person \\\midrule

Nominative & \parbox{1.5cm}{-a{[}m{]} \\ -аn} & -at & -es' & -a{[}m{]} / -аn & -ant & -edes' \\\addlinespace

Genitive & -аn & -at & -es' & -an / -edan & -еdant & -edes' \\\addlinespace

Accusative & -аn & -at & -es' & -e(t)dan & -еdant & -edes' \\\addlinespace

Essive & -jan & -jant & -jas ' & -jedan & -jedant & -jedes' \\\addlinespace
\parbox{2cm}{Inessive-Elative} & -san & -sant & -esan & -esan & -esant & -eses' \\\addlinespace

\parbox{2cm}{Dative-Illative} & -(ja)san & -(je)sant & -jes' & -jedan & -jedant & -jedas' \\\addlinespace

\bottomrule
\end{tabular}
\caption{Part of Possessive Paradigm of Kildin Saami}\label{saami_poss1}
\end{table}

\begin{table}[ht]
\centering
\begin{tabular}{lllllll}
\toprule
\multicolumn{1}{c}{\multirow{2}{*}{Case}} & \multicolumn{3}{c}{\Sg{} possessor, \Pl{} possessum} & \multicolumn{3}{c}{\Pl{} possessor, \Pl{} possessum}\\
\multicolumn{1}{c}{} & 1 person & 2 person & 3 person & 1 person & 2 person & 3 person \\\midrule

Nominative & \parbox{1.5cm}{-a{[}m{]} \\ -аn} & -ant & -edes' & -edan & -edant & -edes' \\\addlinespace

Genitive & -edаn & -edant & -edes' & -edan & -еdant & -edes' \\\addlinespace

Accusative & -edаn & -edant & -edes' & -edan & -еdant & -edes' \\\addlinespace

Essive & -edаn & -jedant & -jedes' & -jedan & -jеdant & -jedes' \\\addlinespace
\parbox{2cm}{Inessive-Elative} & -esan & -esant & -eses't' & -esan & -esan & -eses't' \\\addlinespace

\parbox{2cm}{Dative-Illative} & -ejdan & -jedant & -jedas & -jedan & -jedant & -jedas \\\addlinespace
\bottomrule
\end{tabular}
\caption{Part of Possessive Paradigm of Kildin Saami}\label{saami_poss2}
\end{table}

    In my view, the assertion made by \cite{riesler_kildin_2022} is not entirely accurate. While it is true that the possessive declension is nearly extinct for nouns (as is the case in other Saami languages), it had been grammaticalized in reflexives and reciprocals prior to its loss. Therefore, possessive markers are obligatory in these structures, as demonstrated in example \parref{refl_recip}.

    \pex \label{refl_recip}
    \a \begingl
    \gla nɨzan ujjn-av kaan'n'c' kaan'n'c'-*(es')//
    \glb women see-\Npst.\Tpl{} friend friend-\Poss{}\Third{}//
    \glft `The women see each other.'//
    \endgl
    \a \begingl
    \gla par̥'r̥'š'a oaffk iž'-*(es')//
    \glb boy scolds self-\Poss{}\Third{}//
    \glft `The boy scolds himself.'//
    \endgl
    \xe

    Given that reflexives and reciprocals cannot be formed without the use of possessive morphology, yet possessive morphology is not commonly employed in the language, it is understandable that this presents an intriguing area of study.

    Conventional possessive forms are infrequent and only appear sporadically. Nevertheless, these sporadic instances align with the diversity found in reflexive pronoun formation.

    Speakers of Kildin Saami may be classified into two distinct groups based on the number of person values they differentiate in the possessive declension. One group of speakers differentiates between all three values, while the other group distinguishes only between two. In the latter group, first and second person values are always combined and expressed using a single marker, which consistently appears as \textit{-ant}, the ex-marker for second person. This differentiation is illustrated in \parref{two_types}, with different colors representing each group of speakers.

    \pex\label{two_types}
    \a\begingl
    \gla munn iž'-\textcolor{orange}{an}/-\textcolor{purple}{ant} šobbš-a//
    \glb I self-\textcolor{orange}{\Poss{}\First}/\textcolor{purple}{\Poss{}\First\_\Second{}} love-\Npst{}\Fsg//
    \glft `I like myself.'//
    \endgl

    \a\begingl
    \gla toonn	soagg-ex	kul'	iiǯ-s'-\textcolor{orange}{ant}/\textcolor{purple}{ant}//
    \glb you.\Sg{} catch-\Pst.\Ssg{} fish.\Acc{} self-\textcolor{orange}{\Poss{}\Second{}}/\textcolor{purple}{\Poss{}\First\_\Second{}}//
    \glft `Did you fish for yourself?'//
    \endgl

    \xe

In Kildin Saami, speakers have different preferences regarding which nouns can take possessive suffixes. However, they generally adhere to a hierarchy: kinship terms > domestic animals > certain household items. This hierarchy implies that all speakers allow possessive markers on (some) kinship terms. If a speaker allows possessive marking on certain household items, they typically also allow possessive markers on certain domestic animals. Nouns that do not fit into these categories are less likely to be found with possessive markers.

In Kildin Saami, there is a set of possessive markers that are dependent-marked, known as possessive pronouns. These pronouns have largely replaced possessive suffixes in usage. Speakers of Kildin Saami prefer using these possessive pronouns, with possessive suffixes being rarely employed, except in cases involving reflexives and reciprocals as mentioned earlier.

\subsubsection*{Summary on Kildin Saami}

Kildin Saami has the preference for dependent marked possessive markers, specifically possessive pronouns, over possessive suffixes. These possessive pronouns have largely replace the use of possessive suffixes. Speakers of Kildin Saami commonly use possessive pronouns to indicate possession, with possessive suffixes reserved for specific contexts such as reflexives and reciprocals.

Furthermore, Kildin Saami speakers exhibit a hierarchical pattern in determining which nouns can take possessive markers. This hierarchy prioritizes certain categories of nouns, such as kinship terms, domestic animals, and household items, for possessive marking. Nouns falling outside these categories are less likely to be marked with possessive suffixes or pronouns, and when they are, it occurs sporadically. Concluding, possessive suffixes in Kildin Saami are not obligatory while possessive pronouns are obligatory.

\subsection{Marind}

Marindic < Anim

In Marind, possessive relation can be expressed by two different strategies: a postpositional phrase headed by \textit{en} expressing the possessor \parref{marind_en}, and juxtaposition of the possessor and possessum \parref{marind_just}.

According to \citep[158]{olsson2021grammar}, \textit{en} is dependent marking since it forms a constituent with a possessum.

\ex
\label{marind_en}
\begingl
\gla amay en yay en pula//
\glb ancestor \Poss{} uncle \Poss{} taboo.spot//
\glft `grandpa\'s uncle\'s taboo spot'//
\endgl
\xe

\ex
\label{marind_just}
\begingl
\gla nok onos ɣa k-a-Ø ehe, oɣ ɣakna k-a-Ø//
\glb 1 cousin real aprs.ntrl-3sg-be.npst prox:I 2sg husband's.elder.bro:2sg aprs.ntrl-3sg-be.npst//
\glft `This is my cousin, your brother-in-law.'//
\endgl
\xe

According to \cite{olsson2021grammar}, the postpositional phrase with \textit{en} in Marind has no restrictions on the type of ownership expressed and can convey associative meanings. This construction is flexible in its use and allows for a broad range of possessive relationships to be expressed.

In Marind, many kinship terms have special forms marked with possessive prefixes that are identical to the Undergoer prefixes. While these special marked forms can be used along with the \textit{en} construction, it is not obligatory to do so. Simply using the \textit{en} phrase is sufficient to express possessive relation.


Notably, the juxtaposition strategy of placing the possessor and possessum side by side is not commonly observed with kinship terms in Marind, highlighting a preference for the \textit{en} postpositional phrase when expressing possession involving kinship terms.

\subsubsection*{Summary on Marind}

In Marind, possessive marking involves two distinct sets of markers. The first set utilizes dependent marking with the postposition \textit{en}, which establishes a relationship between the possessor and the possessum within a postpositional phrase. This strategy is versatile and can convey various types of possession and associative meanings.

The second set of markers consists of prefixes occurring on the head of most kinship terms. These special prefixes indicate possession and are closely associated with kinship relationships. The presence of a special prefixed form for a noun implies that the \textit{en} strategy can also be used for possession involving that noun.

However, if there is no special prefixed form available for a noun, speakers of Marind have the option to use either the \textit{en} strategy or simply place the possessor and possessum side by side without any intervening markers. Juxtaposition, in this context, represents a no-marking strategy since Marind lacks case marking.


\printglosses[style=mcolblock]

\printbibliography
\end{document} 
